% ============================================================================
\chapter*{Abstract}
% ============================================================================

The multi-\tev e$^+$e$^-$ Compact Linear Collider (CLIC) is one of the
options for a future high-energy collider for the post-LHC era. It
would allow for searches of new physics and simultaneously offer the
possibility for precision measurements of standard model
processes. The physics goals and experimental conditions at CLIC set
high precision requirements on the vertex detector made of pixel
detectors: a high pointing resolution of $3\,\micron$, very low mass
of $0.2\%$~X\textsubscript{0} per layer, 10~ns time stamping
capability and low power dissipation of 50~mW/cm\textsuperscript{2}
compatible with air-flow cooling. In this thesis, hybrid assemblies
with thin active-edge planar sensors are characterised through
calibrations, laboratory and test-beam measurements. Prototypes
containing $50\,\micron$ to $150\,\micron$ thin planar silicon sensors
bump-bonded to Timepix3 readout ASICs with $55\,\micron$ pitch are
characterised in test beams at the CERN SPS in view of their detection
efficiency and single-point resolution. A digitiser for AllPix, a
\textsc{Geant4}-based simulation framework, has been developed in
order to gain a deeper understanding of the charge deposition spectrum
and the charge sharing in such thin sensors. The AllPix framework is
also used to simulate the beam telescope and extract its tracking
resolution. It is also employed to predict the resolution that can be
achieved with future assemblies with thin sensors and smaller
pitch. For CLIC, a full coverage of the vertex detector is essential
while keeping the material content as low as possible. Seamless tiling
of sensors, without the need for overlaps, by the active-edge
technology allows for extending the detection capability to the
physical edge of the sensor and thereby minimising the inactive
regions. Thin-sensor prototypes containing active edges with different
configurations are characterised in test-beams in view of the
detection performance at the sensor edge. Technology Computer-Aided
Design (TCAD) finite-element simulations are implemented to reproduce
the fabrication and the operation of such devices. The simulation
results are compared to data for different edge terminations.
