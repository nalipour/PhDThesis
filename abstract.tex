% ============================================================================
\chapter*{Abstract}
% ============================================================================

The Compact Linear Collider (CLIC), as a multi-\tev linear e$^+$e$^-$
collider, is considered as one of the options to complement the LHC
experiments at CERN. In the post-LHC era, CLIC will allow for
exploring a great number of searches for new physics and provides
sensitivity for precision physics measurements. The physics goals and
experimental conditions at CLIC, set high precision requirements on
innermost detector: the vertex detector. The principal challenges for
this detector are listed as follows: high pointing resolution of
$3\,\micron$, very low mass of $0.2\%$~X\textsubscript{0} per layer,
10~ns time stamping capabilities of the readout chips, low power
dissipation and compatibility with air-flow cooling and finally power
pulsing operation.

In this thesis, we characterise thin planar sensors with active-edge
technology through calibration, test-beam measurements and
simulations. Hybrid assemblies containing $50-150\,\micron$ thin
planar silicon sensors are bump-bonded to Timepix3 readout ASICs. The
assemblies are integrated within the Timepix3 beam telescope and
characterised during test beams at the CERN SPS. A
\textsc{Geant4}-based simulation framework, AllPix, has been set up in
order to understand the charge deposition spectrum in such thin
sensors and the charge sharing between neighbouring pixels. Data are
as well used to validate the AllPix simulations. The latter offers the
flexibility to integrate any geometry of pixel detectors and simulate
the whole experimental setup such as a test-beam. Therefore, it is
used to calculate the tracking resolution of the Timepix3 beam
telescope. It is also employed to predict the resolution that can be
achieved for a pixel detector of $25\,\micron$ pitch bump-bonded to a
thin sensor of $50\,\micron$ aimed for CLIC, where data is still
unavailable.


For CLIC, a high coverage of the vertex detector is required while
keeping the material budget as low as possible. This prevents overlaps
between pixel detectors while tiling them into ladders. Also, the
detectors must be able to detect up to their physical edges. The
active-edge technology allows for growing the depletion volume up to
the physical edges of silicon sensors and minimise the inactive
regions. The thin-sensor prototypes tested in test-beam also contain
active edges with different configurations. The efficiency of
detection at the edge is characterised in test-beams. Technology
Computer-Aided Design (TCAD) simulations are also implemented to
simulate the fabrication of such devices and their
operation. Simulations are compared to the data for different
configurations.

Finally, conclusions are given based on the collected data and the
different types of simulations performed for thin and active-edge
planar sensors.
