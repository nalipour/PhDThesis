% ============================================================================== 
\chapter*{Résumé}
% ============================================================================== 


Le Collisionneur Linéaire Compact (CLIC), ou le Compact Linear
Collider, est un accélérateur de particules qui permet d'effectuer des
collisions entre des électrons et des positrons à des énergies de
l'ordre de quelques TeV. Le CLIC, est considéré comme l’une des
options qui pourrait être complémentaire aux expériences du LHC pour
la découverte de la physique au-delà du modèle standard et de réaliser
des mesures plus précises sur le boson de Higgs.

Afin d'atteindre les objectifs de découvertes pour CLIC, un détecteur
de vertex fait de détecteurs à pixel de haute précision est en cours
de développement. Une haute résolution spatiale de 3 um, une légère
masse de 0.2\%~X\textsubscript{0} par couche de détection, une
capacité de timestamp de 10 ns, une basse dissipation de puissance de
l’ordre de 50 mW/cm\textsuperscript{2} et compatible avec un
refroidissement à air sont parmi les exigences requises par le
détecteur de vertex pour CLIC. Dans cette étude, nous nous concentrons
sur les assemblages de détecteur à pixel hybride en silicium à bords
mince et actif. Ces assemblages sont caractérisés à travers la
calibration et des mesures dans un faisceau de particules à haute
énergie du SPS au CERN. Ces prototypes contiennent une couche de
détecteur planaire mince avec des épaisseurs qui varient entre 50 um
et 150 um. Ces détecteurs sont interconnectés aux puces de lecture
Timepix3 avec des pixels de taille de 55 um. Des simulations basées
sur Geant4 (AllPix) ont été développées pour une meilleure
compréhension du fonctionnement et de la résolution spatiale de ces
détecteurs. AllPix est aussi employé pour la simulation du telescope
de faisceau et ainsi extraire ses capacités pour la reconstruction des
traces. Ces simulations servent aussi à prédire la résolution que
pourraient atteindre les détecteurs futurs minces avec des pixels plus
petits. Pour le CLIC, une couverture intégrale du détecteur de vertex
est essentielle tout en minimisant le contenu matériel. Les détecteurs
à bords minces permettent de couvrir intégralement les surfaces du
détecteur à vertex en réduisant les zones inactives sans créer des
chevauchements entre les modules des détecteurs à pixel. Des
prototypes, avec des configurations différentes aux bords, sont testés
en test faisceau de haute énergie afin de qualifier l’efficacité du
bord actif. Des simulations TCAD, basées sur une méthode des éléments
finis, sont mises en place pour reproduire la fabrication et
l’opération de ces dispositifs. Les mesures sont comparées avec le
résultat des simulations.


% Le Collisionneur Linéaire Compact (CLIC), ou le Compact Linear
% Collider, est un accélérateur de particules qui permet d'effectuer des
% collisions entre des électrons et des positrons à des énergies de
% l'ordre de quelques \tev. Le CLIC, est considéré comme une des options
% qui pourrait être complémentaire aux expériences du LHC pour la
% découverte de la physique au-delà du modèle standard et de réaliser
% des mesures plus précises sur le boson de Higgs par exemple.

% Un détecteur de haute précision est en cours de développement afin
% d'atteindre les objectifs de découvertes pour CLIC. Cette thèse se
% concentre sur la conception du sous-détecteur le plus proche du point
% d'interaction des particules: le détecteur de vertex qui sera réalisé le pixel
% détecteur à silicium pour
