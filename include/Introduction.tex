% ==============================================================================             
\chapter{Introduction}
\label{sec:intro}
% ==============================================================================             

The Compact Linear Collider (CLIC) concept~\cite{Aicheler:1500095} for
a future linear e\textsuperscript{+}e\textsuperscript{-} collider is
under development by the international CLIC collaboration. Its physics
programme has the potential to complement the measurements done by the
LHC experiments. With proposed centre-of-mass energies of $380\,\gev$,
$1.5\,\tev$ and $3\,\tev$ and with an instantaneous luminosity up to
$6\times10^{34}\,\inversecmsquaredsec$, this lepton collider allows
for high precision measurements of Standard Model physics and of new
physics potentially discovered at the $13\,\tev$ LHC as well as
searches for Beyond Standard Model (BSM) physics.

For the CLIC experiment, a detector is under development which takes
into account the precision physics requirements and experimental
conditions~\cite{Linssen:1425915}. The CLIC detector, like most of the
detectors in high-energy physics experiments, consists of several
sub-detectors. The innermost sub-detector, which is the closest to the
interaction point, is a silicon pixel detector with three double-sided
layers in the barrel region and a spiral geometry in the endcap
region. Its main goal is to distinguish heavy quarks from
light-flavoured quarks through a precise measurement of their
displaced decay vertices. The precision physics requirements set
challenging demands for the vertex detector in terms of spatial
resolution, material budget with efficient heat removal from sensors
and readout and also, timing resolution. The CLIC vertex detector R\&D
programme studies different detector technologies and takes into
account constraints from mechanics, power delivery and cooling. In
order to reduce the multiple scatterings and achieve high precision on
measurements, a low total material budget of
$\sim0.2\%$~X\textsubscript{0} per vertex detector layer is required
including readout, support and cabling. The goal is to achieve a
single-point resolution of $3\,\micron$ with $50\,\micron$ thick
sensors coupled to $50\,\micron$ thick readout ASICs with
$25\,\micron$ pixel pitch.

In this thesis, the feasibility of thin pixelated planar silicon
sensors for operation in the CLIC vertex detector is
studied. Assemblies with $50\,\micron$ to $150\,\micron$ thick sensors
are bump bonded to Timepix3 readout ASICs~\cite{Timepix3Poikela} with
a pixel size of $55\,\micron$. These assemblies are tested during
test-beam campaigns at the CERN SPS. Data are used for the
charcterisation of the thin sensors in terms of energy deposition,
charge sharing and spatial
resolution. \textsc{Geant4}-based~\cite{Agostinelli:2002hh}
simulations are used for a better understanding of thin sensors and
validated with data. The simulation is then extrapolated for smaller
pixels of $25\,\micron$ where data are not fully available. This gives
a hint on the spatial resolution which can be achieved for the CLIC
vertex detector using a planar sensor technology.

Active-edge sensors allow for seamless tiling of pixel sensors by
depleting the sensors up to their physical edges. In a vertex
detector, this allows for high coverage without creating overlaps
between the pixel sensors and therefore reduces the material
content. Measurements of prototypes and corresponding finite-element
TCAD simulations are performed to compare the efficiency of different
active-edge sensor layouts, resulting in a proposal for a suitable
layout.

% Different guard
% ring solutions for thin active-edge sensors are considered for the
% CLIC vertex detector. Prototypes have been characterised with
% measurements and simulations and an optimal solution is proposed.

This thesis is structured as follows. \cref{ch:CLIC} describes the
CLIC experiment. The accelerator is briefly introduced. The
requirements on the CLIC detector, with a focus on the vertex
detector, are given. The flavour-tagging performance is studied for
different geometries of the vertex detector.

As this thesis focuses on the R\&D of silicon detectors, principles of
the semiconductor detectors are presented in \cref{sec:SiliconTheory}.

The Timepix3 readout ASIC is presented in
\cref{ch:FE_electronics}. The tested assemblies are introduced and
their readout noise is measured. The calibration methods for these
assemblies are discussed.

The simulation and reconstruction software frameworks are described in
\cref{ch:Software}. The reconstruction frameworks are used for the
analysis of simulations and data.

The Timepix3 pixel beam-telescope used for testing the Timepix3
assemblies is described in \cref{ch:Telescope}. Its performance is
investigated in data and simulations.

The feasibility of operating and the performance of thin sensors are
studied in \cref{ch:ThinSensorsStudies}. Simulations are validated
with test-beam data. The spatial resolution for a $25\,\micron$ pitch
with a $50\,\micron$ thick pixel detector is extrapolated using the
simulations.

The performance of active-edge sensors is discussed in
\cref{ch:ActiveEdgeSensors}. Different designs for the guard ring at
the edge are considered. Test-beam results are compared to TCAD
simulations.

\cref{ch:conclusions} summarises the obtained results with conclusions
and outlook.

% Different guard ring solutions for thin active-edge sensors and their
% test beam results are introduced in
% Section~\ref{sec:SectionActiveEdge}.

% CLICpix~\cite{clicpix}, a readout ASIC with $25\,\micron$
% pixel pitch, is developed in 65~nm CMOS technology. CLICpix is either
% bump bonded to planar silicon sensors or capacitively coupled through a thin
% layer of glue to active sensors implemented in a commercial 180~nm
% High-Voltage (HV) CMOS process~\cite{AlipourTehrani:2048684}.  Results of
% recent test beam measurements for these two hybridisation concepts are
% presented in Section~\ref{sec:SectionCLICpix}.
