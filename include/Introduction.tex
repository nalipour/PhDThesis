% ==============================================================================             
\chapter{Introduction}
\label{sec:intro}
% ==============================================================================             

The Compact Linear Collider (CLIC) concept~\cite{Aicheler:1500095} for
a future linear e\textsuperscript{+}e\textsuperscript{-} collider is
under development by the international CLIC collaboration. Its physics
programme has the potential to complement the measurements done by the
LHC experiments. With proposed centre-of-mass energies of $380\,\gev$,
$1.5\,\tev$ and $3\,\tev$ and with an instantaneous luminosity up to
$6\times10^{34}\,\inversecmsquaredsec$, this lepton collider allows
for high precision measurements of Standard Model physics and of new
physics potentially discovered at the $13\,\tev$ LHC, as well as
searches for Beyond Standard Model (BSM) physics.

For the CLIC experiment, a detector system is under development which
takes into account the precision physics requirements and experimental
conditions~\cite{Linssen:1425915}. The innermost sub-detector, located
closest to the interaction point, is a silicon pixel detector with
three double layers both in the barrel and the endcap regions. Its
main goal is to distinguish heavy quarks from light-flavoured quarks
through a precise measurement of their displaced decay vertices. The
precision physics requirements set challenging demands for the vertex
detector in terms of spatial resolution, material budget with
efficient heat removal from sensors and readout and also timing
resolution. The CLIC vertex detector R\&D programme studies different
detector technologies and takes into account constraints from
mechanics, power delivery and cooling. In order to reduce multiple
scatterings to profit from the good impact parameter resolution, a low
total material budget of $\sim0.2\%$~X\textsubscript{0} per vertex
detector layer is required including readout, support and cabling. The
goal is to achieve a single-point resolution of $3\,\micron$ with
$50\,\micron$ thick sensors coupled to $50\,\micron$ thick readout
ASICs with $25\,\micron$ pixel pitch.

In this thesis, the feasibility of thin pixelated planar silicon
sensors for operation in the CLIC vertex detector is studied. Planar
silicon technology is already well known and widely used in the pixel
detectors of the modern high-energy physics experiments. Assemblies
with $50\,\micron$ to $150\,\micron$ thick sensors are bump bonded to
Timepix3 readout ASICs~\cite{Timepix3Poikela} with a pixel size of
$55\,\micron$. These assemblies are tested during test-beam campaigns
at the CERN SPS. Data are used for the charcterisation of the thin
sensors in terms of energy deposition, charge sharing and spatial
resolution. A \textsc{Geant4}-based~\cite{Agostinelli:2002hh}
simulation of the signal formation in planar silicon sensors has been
developed and is used to gain a better understanding of thin
sensors. After being validated with data, the simulation is used to
investigate the possible performance of sensors with smaller pixels
where no experimental data is yet available.

Active-edge sensors allow for seamless tiling of pixel sensors by
depleting the sensors up to their physical edges. In a vertex
detector, this allows for high coverage without creating overlaps
between the pixel sensors and therefore reduces the material
content. Efficiency measurements in test-beams on prototypes and
corresponding finite-element TCAD simulations are performed to compare
the performance of different active-edge sensor layouts, resulting in
a proposal for a suitable layout.

% Different guard
% ring solutions for thin active-edge sensors are considered for the
% CLIC vertex detector. Prototypes have been characterised with
% measurements and simulations and an optimal solution is proposed.

This thesis is structured as follows. \cref{ch:CLIC} describes the
CLIC experiment. The accelerator concept based on a novel 2-beam
acceleration scheme is briefly introduced. The requirements on the
CLIC detector, with a focus on the vertex detector, are given. The
flavour-tagging performance is studied for different geometries of the
vertex detector.

General working principles of the semiconductor detectors are
presented in \cref{sec:SiliconTheory}.

The Timepix3 readout ASIC is introduced in
\cref{ch:FE_electronics}. The tested assemblies are presented and
their readout noise is measured. The calibration methods for these
assemblies are discussed.

The simulation and reconstruction software frameworks are described in
\cref{ch:Software}.

The Timepix3 pixel beam-telescope used for testing the Timepix3
assemblies is described in \cref{ch:Telescope}. Its performance is
investigated in data and simulations.

The performance of thin sensors is studied in
\cref{ch:ThinSensorsStudies} and simulations are validated with
test-beam data. The spatial resolution of a $50\,\micron$ thin
detector with $25\,\micron$ pitch (as well as smaller pixel pitches)
is estimated using the simulation model.

The performance of active-edge sensors is discussed in
\cref{ch:ActiveEdgeSensors}. Different designs for the guard ring at
the edge are considered and the results obtained in the test beams are
compared to TCAD simulations.

\cref{ch:conclusions} summarises the obtained results and gives
conclusions.

% Different guard ring solutions for thin active-edge sensors and their
% test beam results are introduced in
% Section~\ref{sec:SectionActiveEdge}.

% CLICpix~\cite{clicpix}, a readout ASIC with $25\,\micron$
% pixel pitch, is developed in 65~nm CMOS technology. CLICpix is either
% bump bonded to planar silicon sensors or capacitively coupled through a thin
% layer of glue to active sensors implemented in a commercial 180~nm
% High-Voltage (HV) CMOS process~\cite{AlipourTehrani:2048684}.  Results of
% recent test beam measurements for these two hybridisation concepts are
% presented in Section~\ref{sec:SectionCLICpix}.
