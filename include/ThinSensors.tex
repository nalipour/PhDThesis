% ==============================================================================
\chapter{Thin sensors studies}
\label{ch:ThinSensorsStudies}
% ==============================================================================    

The aim of the CLIC vertex detector is the precise determination of
the displaced vertices and jet flavour-tagging in an environment with
high occupancy from beam-induced backgrounds. Multi-layer barrel and
endcap pixel detectors with geometrical coverage extending down to low
polar angles ($\theta_{min}\sim8^{\circ}$) are foreseen to fulfill
these requirements. The goal is to achieve a single-point resolution
of $\sim 3\,\micron$ with $25\,\micron$ pixel pitch and analog
readout. A material budget of $\sim0.2\%$~X\textsubscript{0} per layer
is required including readout, support and cabling. This can be
achieved with $50\,\micron$ thick sensors on $50\,\micron$ thick
readout ASICs.

As part of the R\&D programme for the CLIC vertex detector, the
performance of thin silicon sensors is quantified using the Timepix3
readout chip with $55\,\micron$ pixel pitch. $50\,\micron$ to
$300\,\micron$ thick planar sensors are bump-bonded to $700\,\micron$
thick Timepix3 ASICs. These assemblies are tested in test beams and
the data are used to compare to \textsc{Geant4}-based simulations. The
simulation is finally extrapolated for smaller pixels of
$25\,\micron$ pixel pitch.

%% --------------------------------------------- %%
\section{Thin-sensor assemblies}
A summary of the Timepix3 assemblies bump-bonded to thin sensors is
shown in \cref{tab:Timepix3Assemblies}. 

The assemblies were tested in beam reference telescope as shown in
\cref{fig:TPX3Telescope}. The assemblies were placed in the middle of
the Timepix3 telescope.

%% --------------------------------------------- %%
\subsection{Operating conditions}\label{sec:operatingConditions}
The Timepix3 readout ASICs were operated with the optimised parameters
as described in \cref{tab:timepix3Operation}. 

The nominal values for the threshold and the bias voltage are shown in
\cref{tab:nominalBiasThreshold}. For the nominal bias voltage, the
sensor is fully (or over) depleted. The nominal threshold set insures
that the readout chip is not operating in noisy conditions. These
nominal values are held throughout the data taking except for the
cases where the threshold or the bias voltages are scanned.

\begin{table}[htbp]
  \centering
  \caption{Nominal operating bias voltage, threshold in DAC and
    calibrated in number of electrons (measured as described in
    \cref{sec:thresholdCalibration}) for the tested assemblies as
    shown in \cref{tab:Timepix3Assemblies}.}
  \label{tab:nominalBiasThreshold}
  \begin{tabular}{lccc}
    \toprule
    Timepix3 ID & Bias voltage [V] & Threshold [DAC] & Threhsold [e\textsuperscript{-}]\\
    \midrule
    W19\_G7 & -15 & 1190 & $526.230\pm0.649$ \\
    W19\_F7 & -15 & 1187 & $600.406\pm0.551$ \\
    W19\_L8 & -15 & 1133 & $568.506\pm0.538$ \\
    W19\_C7 & -15 & 1148 & $608.700\pm0.488$ \\ \hline
    W5\_E2 & -20 & 1160 & $561.019\pm0.712$ \\ \hline
    W5\_F1 & -30 & 1153 & $554.885\pm0.493$ \\ \hline
    W2\_J5 & 100 & 1170 & $565.723\pm1.560$ \\
    \bottomrule
  \end{tabular}
\end{table}

% %% --------------------------------------------- %%
% \section{Samples and sensors geometries}
% %% The DAC settings used for the operation of the Timepix3 assemblies in
% %% test beams is summarised in \cref{tab:timepix3Operation}.
% %% \begin{itemize}
% %% \item I\textsubscript{krum} DAC is set to 10.
% %% \item TOT clock frequency: $40\,\megahertz$
% %% \item VFBK: 150
% %% \end{itemize}


% For most of the measurements, the DUT is perpendicular to the beam. A
% scan was done on the bias voltage and the threshold of the DUT. The
% bias scan allows to obtain the depletion voltage.

%% --------------------------------------------- %%
\section{Experimental results for thin sensors}

\subsection{Measurement of the depletion voltage}

The depletion voltage corresponds to the voltage at which the sensor
is fully depleted and the charge of the full thickness of the sensor
is collected. This can be measured in test beams by scanning the bias
voltage. Lowering the bias voltage causes the under-depletion of the
sensor and therefore the charge collected is also reduced. The
depletion voltage and the depletion width are related by
\cref{eq:depletionVoltage}. 

\cref{fig:W2_J5_DepletionVoltage_chargeVSbiasVoltage} shows the most
probable value (MPV) of the measured TOT as a function of the square
root of the bias voltage. The measured TOT is obtained by summing up
the TOT in the clusters considering all cluster sizes. The
distribution is fitted with a landau function convoluted with a
Gaussian. The fit is done using RooFit~\cite{Cranmer:2012sba} and an
example is given in \cref{fig:W2_J5_DepletionVoltage_TOTdistr}. The
TOT MPV shows two distinct regions: a sloped region and a plateau
region. The intersection of these two regions gives the depletion
voltage of the assembly and it is calculated by the intersection of a
linear fits performed on the both regions. The calculated depletion
voltages for all the assemblies is presented in
\cref{tab:depletionVoltage} and shown in \cref{fig:depletionVoltage}.

\begin{figure}[htbp]\centering
  \begin{subfigure}[b]{0.45\textwidth}
  \includegraphics[width=\textwidth]{./figures/TestBeam/W2_J5_totalTOT_Langau_run1998.pdf}
  \caption{}\label{fig:W2_J5_DepletionVoltage_TOTdistr}
  \end{subfigure} \hfill
  \begin{subfigure}[b]{0.45\textwidth}
    \includegraphics[width=\textwidth]{./figures/TestBeam/depletionVoltage_W0002_J05.pdf}
    \caption{}\label{fig:W2_J5_DepletionVoltage_chargeVSbiasVoltage}
  \end{subfigure}
  \caption{(a) The measured TOT distribution in a $300\,\micron$ thick
    silicon sensor (W2\_J5) with a bias voltage of 95~V fitted with a
    Landau function convoluted with a Gaussian function. (b) The most
    probable value of the measured TOT as a function of the bias
    voltage for W2\_J5. Straight lines are used to fit the slope and
    the plateau regions. The depletion voltage corresponds to the
    intersection of these two regions and shown in a red dashed
    line. The continuous red line shows the nominal operating bias
    voltage.}
  \label{fig:W2_J5_DepletionVoltage}
\end{figure}

\begin{table}[htbp]
  \centering
  \caption{Measured depletion voltage for the assemblies described in
    \cref{tab:Timepix3Assemblies} and calculated by fitting the
    plateau and slope regions of TOT as a function of bias voltage.}
  \label{tab:depletionVoltage}
  \begin{tabular}{lc}
    \toprule
    Assembly & Depletion voltage [V] \\
    \midrule
    W19\_G7 & $<$-9.47 \\
    W19\_F7 & $<$-6.32 \\
    W19\_L8 & $<$-7.19\\
    W19\_C7 & $<$-5.43\\ \hline
    W5\_E2 & -10.82 \\ \hline
    W5\_F1 & -14.86 \\ \hline
    W2\_J5 & 63.23 \\ 
    \bottomrule
  \end{tabular}
\end{table}


\subsection{Charge sharing as a function of the track position}

The charge sharing is studied by extrapolating the track position
obtained from the telescope to the DUT. The size of the clusters
depends on the track position within the pixel. For the perpendicular
tracks to the DUT, multi-pixel clusters are created when the track
hits the edges or the corners of a pixel, leading to the charge
sharing between the pixel and its neighbouring pixels. The charge
sharing is also affected by the operating threshold of the readout
ASIC in addition to the track position. \cref{fig:chargeSharingTrack}
illustrates the track position within the pixel for 1 to 4-pixel cluster
sizes for a $50\,\micron$ sensor. For other thicknesses, the track
position within the pixel is shown in
\cref{fig:chargeSharingTrack_W19_G7,fig:chargeSharingTrack_W5_E2,fig:chargeSharingTrack_W5_F1}.

\begin{figure}[htbp] \centering
  \begin{subfigure}[b]{0.23\textwidth}
    \includegraphics[width=\textwidth]{./figures/TestBeam/TrackPosWPixel_1hit_runW19_G7.pdf}
    \caption{Cluster size 1}
  \end{subfigure} \hfill
  \begin{subfigure}[b]{0.23\textwidth}
    \includegraphics[width=\textwidth]{./figures/TestBeam/TrackPosWPixel_2hit_runW19_G7.pdf}
    \caption{Cluster size 2}
  \end{subfigure} \hfill
  \begin{subfigure}[b]{0.23\textwidth}
    \includegraphics[width=\textwidth]{./figures/TestBeam/TrackPosWPixel_3hit_runW19_G7.pdf}
    \caption{Cluster size 3}
  \end{subfigure} \hfill
  \begin{subfigure}[b]{0.23\textwidth}
    \includegraphics[width=\textwidth]{./figures/TestBeam/TrackPosWPixel_4hit_runW19_G7.pdf}
    \caption{Cluster size 4}
  \end{subfigure}
  \caption{Extrapolated track position within the pixel for 1 to 4-pixel
    cluster sizes for a $50\,\micron$ sensor (assembly W19\_G7). The
    assembly is operated at the nominal conditions.}
  \label{fig:chargeSharingTrack}
\end{figure}

\cref{fig:chargeSharing_2PC} compares the track position within the
pixel for two-pixel clusters in a one-dimensional profile for
different sensor thicknesses. The thinner the sensor, the track
position is closer to the edges of the pixels to create a two-pixel
cluster.

\begin{figure}[htbp] 
  \centering
  \includegraphics[width=0.5\textwidth]{./figures/TestBeam/chargeSharing_2pixel_clusters.pdf}
  \caption{Track position within the pixel for tracks leading to
    two-pixel clusters for different sensor thicknesses (the
    histograms are scaled to have a unit area).}
  \label{fig:chargeSharing_2PC}
\end{figure}

\subsection{Cluster size distribution}
The cluster size distribution is a straight indication on the amount
of charge sharing in a sensor. For thicker sensors, more charge
sharing is expected since the charge drifts in a longer distance and
diffuses across a larger transverse area. The threshold of the readout
electronics has a large impact on the detection of the charge
sharing. The threshold should be as low as possible to avoid
undetected energy deposits.

In this analysis, the hit pixels form a cluster with a distance
criterion of $\sqrt{2}$: pixels sharing common corners or edges are
considered as clusters. For sensor thicknesses in the range of
$50\,\micron$-$300\,\micron$, clusters of size one to four are
expected due to the charge sharing. Larger clusters are due to other
factors such as the $\delta$-rays.

The fraction of different cluster sizes depends on the operating
conditions of the assemblies like the bias voltage and the threshold
of the readout ASICs. \cref{fig:cluSize_operatingConditions} shows the
cluster size fraction for a $150\,\micron$ thick sensor (assembly
W5\_F1) operated at different bias voltages and thresholds.  

Higher bias voltages reduce the amount of charge sharing since the
drift time is lowered in a stronger electric field and the charges
diffuse less in the transverse direction. For very low bias voltages,
the sensor is under-depleted. An increase in bias voltage in this
regime increases the charge sharing. In fact, the pixel neighbouring a
hit pixel is more likely to be over threshold as more charge is
collected. The results for other assemblies are shown in
\cref{fig:clusterSize_vs_biasVoltage}.

Lowering the threshold leads to an increase in the cluster size
fraction distribution and thus a higher detection of the charge
sharing. Results for all the assemblies are shown in
\cref{fig:clusterSize_vs_THLscan}.


\begin{figure}[htbp] \centering
  \begin{subfigure}[b]{0.45\textwidth}
    \includegraphics[width=\textwidth]{./figures/TestBeam/cluSize_biasScan_W0005_F01.pdf}
    \caption{}
  \end{subfigure} \hfill
  \begin{subfigure}[b]{0.45\textwidth}
    \includegraphics[width=\textwidth]{./figures/TestBeam/cluSize_THLscan_W0005_F01.pdf}
    \caption{}
  \end{subfigure}
  \caption{Fraction of cluster sizes as a function of (a) bias voltage
    and (b) threshold for the assembly W5\_F1 with a $150\,\micron$
    thick sensor.}
  \label{fig:cluSize_operatingConditions}
\end{figure}


The cluster size fraction as a function of the sensor thickness for
data recorded at the nominal operating conditions is shown in
\cref{fig:cluSize_thickness}. Charge sharing increases with sensor
thickness. The fraction of $1\times2$ and $2\times1$-pixel clusters are
very similar for square pixels and 2-pixel clusters combine these two
cluster geometries.

\begin{figure}[htbp] 
  \centering
  \includegraphics[width=0.5\textwidth]{./figures/TestBeam/cluSize_vs_thickness.pdf}
  \caption{}
  \label{fig:cluSize_thickness}
\end{figure}

\subsection{Single-point resolution}

From the test-beam data, the residuals are calculated by comparing the
reconstructed hit position with the extrapolated track position
obtained by the telescope. This residual combines in quadrature the
single-point (or hit) and the track resolutions:

\begin{equation}
  \sigma_{\mathrm{residual}}^{2}=\sigma_{\mathrm{hit}}^{2}+\sigma_{\mathrm{track}}^{2} .
  \label{eq:residualEq}
\end{equation}

In the following, the residual measurements (as defined by
\cref{eq:residualEq}) are presented in the Figures. For the estimation
of the single-point resolution, the tracking resolution of
$\sim2\,\micron$ at the SPS test beam should be unfolded.

The residuals depend highly on the cluster sizes since the
reconstructed hit position takes into account the number of the hit
pixels in a cluster and the charge information in each
pixel. \cref{fig:residuals_cluSize} illustrates the residuals for the
a $50\,\micron$ thick sensor (assembly W19\_G7) for different cluster
sizes. For single-pixel clusters, the hit position corresponds to the
geometric center of the pixel and therefore a resolution of pixel
pitch$/\sqrt{12}$ is expected (as explained in
\cref{sec:binaryReadout}). For multi-pixel clusters, by using the
charge information, a more accurate interpolation of the hit position
can be obtained and the resolution is better than the geometrical
information (by using the $\eta$-correction method as described in
TODO???).

\begin{figure}[htbp] \centering
  \begin{subfigure}[b]{0.23\textwidth}
    \includegraphics[width=\textwidth]{./figures/TestBeam/residual_1hit_W19_G7.pdf}
    \caption{Cluster size 1}
  \end{subfigure} \hfill
  \begin{subfigure}[b]{0.23\textwidth}
    \includegraphics[width=\textwidth]{./figures/TestBeam/residual_2hit_W19_G7.pdf}
    \caption{Cluster size 2}
  \end{subfigure} \hfill
  \begin{subfigure}[b]{0.23\textwidth}
    \includegraphics[width=\textwidth]{./figures/TestBeam/residual_3hit_W19_G7.pdf}
    \caption{Cluster size 3}
  \end{subfigure} \hfill
  \begin{subfigure}[b]{0.23\textwidth}
    \includegraphics[width=\textwidth]{./figures/TestBeam/residual_4hit_W19_G7.pdf}
    \caption{Cluster size 4}
  \end{subfigure}
  \caption{The residuals for a $50\,\micron$ thick sensor (assembly
    W19\_G7) for different cluster sizes: (a) Cluster size 1
    ($1\times1$), (b) Cluster size 2 ($2\times1$), (c) Cluster size 3
    ($2\times2$) and (d) Cluster size 4 ($2\times2$). The track
    resolution is not unfolded.}
  \label{fig:residuals_cluSize}
\end{figure}

The overall residual is defined by the average of the residual of each
cluster size weighted by the relative fractions of different cluster
sizes. The factors affecting the fraction of different cluster sizes
such as the sensor thickness and the operating conditions of the
assembly (threshold and bias voltage) affect as well the
residuals. \cref{fig:Residuals_bias_threshold} shows the RMS of the
residuals in the x and y directions for a $50\,\micron$ thick sensor
(assembly W19\_G7). For other assemblies, these plots are shown in
\cref{fig:Residuals_vs_biasVoltage,fig:Residuals_vs_Threshold}. The
residuals follow the same trend as the cluster size distributions:
higher fractions of single-hit clusters result in higher residuals.


\begin{figure}[htbp] \centering
  \begin{subfigure}[b]{0.45\textwidth}
    \includegraphics[width=\textwidth]{./figures/TestBeam/W19_G7_Residual_vs_bias.pdf}
    \caption{}
  \end{subfigure} \hfill
  \begin{subfigure}[b]{0.45\textwidth}
    \includegraphics[width=\textwidth]{./figures/TestBeam/residuals_W0019_G07_THLscan.pdf}
    \caption{}
  \end{subfigure}
  \caption{The RMS of the residuals in x and y directions as a
    function of (a) bias voltage and (b) threshold for a $50\,\micron$
    thick sensor (assembly W19\_G7). The track resolution is not
    unfolded.}
  \label{fig:Residuals_bias_threshold}
\end{figure}


For the assemblies operated at the nominal operating conditions, the
RMS of the residuals in the x and y directions are shown in
\cref{fig:residuals_thickness}. For thicker sensors the fraction of
multi-pixel clusters is higher and results in lower residuals. The
distribution of the residuals for different sensor thicknesses
operated at the nominal conditions is shown in
\cref{fig:residualsHist_thickness}. The wider component in the
residual distributions corresponds to the single-pixel clusters. For
thicker sensors, the fraction of multi-pixel clusters increases. This
leads to a higher fraction of more precisely reconstructed hits and
therefore the residual distribution gets narrower.

\begin{figure}[htbp] 
  \centering
  \includegraphics[width=0.5\textwidth]{./figures/TestBeam/residuals_vs_thickness.pdf}
  \caption{The RMS of the residuals in x and y directions as a
    function of the sensor thickness for assemblies operated at the
    nominal condition. The track resolution is not unfolded.}
  \label{fig:residuals_thickness}
\end{figure}


\begin{figure}[htbp] \centering
  \begin{subfigure}[b]{0.23\textwidth}
    \includegraphics[width=\textwidth]{./figures/TestBeam/residualsHist_W19_C7.pdf}
    \caption{$50\,\micron$ sensor}
  \end{subfigure} \hfill
  \begin{subfigure}[b]{0.23\textwidth}
    \includegraphics[width=\textwidth]{./figures/TestBeam/residualsHist_W5_E2.pdf}
    \caption{$100\,\micron$ sensor}
  \end{subfigure} \hfill
  \begin{subfigure}[b]{0.23\textwidth}
    \includegraphics[width=\textwidth]{./figures/TestBeam/residualsHist_W5_F1.pdf}
    \caption{$150\,\micron$ sensor}
  \end{subfigure} \hfill
  \begin{subfigure}[b]{0.23\textwidth}

    \caption{$300\,\micron$ sensor}
  \end{subfigure}
  \caption{Examples of the residuals distributions in the x direction
    for the assemblies (a) W19\_C7, (b) W5\_E2 and (c) W5\_F1. The
    track resolution is not unfolded.}
  \label{fig:residualsHist_thickness}
\end{figure}
%% --------------------------------------------- %%
\subsection{Global detection efficiency}

The detection efficiency of the assemblies is defined as the fraction
of total number of hits matched to tracks (within a window of radius
0.1~mm) and the total number of tracks projected to pass through the
assembly. The efficiency is calculated within the matrix of
$256\times256$ pixels of the main sensor area and the hot or masked
pixels are not excluded from this calculation. They contribute to the
assemblie's inefficiencies.

The detection efficiency is strongly related to the operating
threshold of the readout ASIC. With a lower threshold, smaller energy
depositions can be measured and it is more likely to detect a
track. \cref{fig:efficiency_VS_Threshold} shows the detection
efficiency for different sensor thicknesses. For thinner sensors, the
energy deposition is lower and the efficiency drops more quickly by
increasing the threshold than for thick sensors.

\begin{figure}[htbp] 
  \centering
  \includegraphics[width=0.5\textwidth]{./figures/TestBeam/Efficiency_vs_THL.pdf}
  \caption{Global detection efficiency as a function of the threshold
    of the readout ASIC for different sensor thicknesses.}
  \label{fig:efficiency_VS_Threshold}
\end{figure}



%% --------------------------------------------- %%
\section{Calibrated test beam data}

The test pulse calibration as described in
\cref{sec:EnergyCalibration} is applied to the data from test beams in
order to convert the energy deposited by the MIP
particles. \cref{sec:testBeamDataCalibrated_vs_G4} compares the
calibrated data with the \textsc{Geant4} energy deposition using the
PAI physics list (c.f. \cref{sec:Silicon_Geant4}) for $50\,\micron$,
$100\,\micron$ and $150\,\micron$ thick planar sensors. There is a
good agreement in terms of the most probable value (MPV) and the
full-width-at-half-maximum (FWHM) in both simulation and data.

\begin{figure}[htbp] \centering
  \begin{subfigure}[b]{0.33\textwidth}
    \includegraphics[width=\textwidth]{./figures/Calibration/Edep_G4_W0019_G07.pdf}
    \caption{55-GNDGR}
  \end{subfigure} \hfill
  \begin{subfigure}[b]{0.33\textwidth}
    \includegraphics[width=\textwidth]{./figures/Calibration/Edep_G4_W0005_E02.pdf}
    \caption{55-GNDGR-100}
  \end{subfigure}\hfill
  \begin{subfigure}[b]{0.33\textwidth}
    \includegraphics[width=\textwidth]{./figures/Calibration/Edep_G4_W0005_F01.pdf}
    \caption{55-GNDGR-150}
  \end{subfigure}
  \caption{Calibrated energy distribution of test beam data for
    different assemblies. The pixel-by-pixel calibration is applied to
    the data which is obtained using test pulses. \textsc{Geant4}
    energy deposition is obtained using the PAI physics list.}
  \label{sec:testBeamDataCalibrated_vs_G4}
\end{figure}

\cref{sec:testBeamDataCalibrated_TOT} shows the TOT distribution for
one, two, three and four-hit clusters. The MPV of the TOT
distributions for different cluster sizes do not align due to the
non-linear behaviour of the Timepix3, as sharing the same deposited
charge amongst several pixels results in each pixel having a higher
TOT than would be expected from simply scaling the charge. The
calibrated energy distribution aligns for different cluster sizes
since the calibration takes into account the non-linearities of the
chip as shown in \cref{sec:testBeamDataCalibrated_Edep}. For the
$50\,\micron$ thick sensor, the calibrated distributions do not
align. This is due to the low energy deposition in thin sensors
especially for multi-hit clusters where the energy deposition per
pixel is close to the threshold.

\begin{figure}[htbp] \centering
  \begin{subfigure}[b]{0.33\textwidth}
    \includegraphics[width=\textwidth]{./figures/Calibration/TOT_Clusters_W0019_G07.pdf}
    \caption{55-GNDGR}
  \end{subfigure} \hfill
  \begin{subfigure}[b]{0.33\textwidth}
    \includegraphics[width=\textwidth]{./figures/Calibration/TOT_Clusters_W0005_E02.pdf}
    \caption{55-GNDGR-100}
  \end{subfigure}\hfill
  \begin{subfigure}[b]{0.33\textwidth}
    \includegraphics[width=\textwidth]{./figures/Calibration/TOT_Clusters_W0005_F01.pdf}
    \caption{55-GNDGR-150}
  \end{subfigure}
  \caption{TOT distribution for one, two, three and four-hit clusters
    for various assemblies with (a) $50\,\micron$, (b) $100\,\micron$
    and (c) $150\,\micron$ thick planar sensors.}
  \label{sec:testBeamDataCalibrated_TOT}
\end{figure}

\begin{figure}[htbp] \centering
  \begin{subfigure}[b]{0.33\textwidth}
    \includegraphics[width=\textwidth]{./figures/Calibration/Edep_Clusters_W0019_G07.pdf}
    \caption{55-GNDGR}
  \end{subfigure} \hfill
  \begin{subfigure}[b]{0.33\textwidth}
    \includegraphics[width=\textwidth]{./figures/Calibration/Edep_Clusters_W0005_E02.pdf}
    \caption{55-GNDGR-100}
  \end{subfigure}\hfill
  \begin{subfigure}[b]{0.33\textwidth}
    \includegraphics[width=\textwidth]{./figures/Calibration/Edep_Clusters_W0005_F01.pdf}
    \caption{55-GNDGR-150}
  \end{subfigure}
  \caption{Energy deposition distribution for one, two, three and
    four-hit clusters for various assemblies with (a) $50\,\micron$,
    (b) $100\,\micron$ and (c) $150\,\micron$ thick planar sensors.}
  \label{sec:testBeamDataCalibrated_Edep}
\end{figure}

For all the assemblies, the energy deposition in \ev is given in \cref{sec:appendixCalibDataG4}.


%% --------------------------------------------- %%
\section{Validation of the simulation}

The AllPix simulation framework (described in \cref{sec:AllPix}) is
used to simulate the performance of thin sensors and to understand the
charge sharing in such sensors. The simulations are compared and
validated with the test-beam data.

The cluster size, the residuals and the energy deposition
distributions are shown in
\cref{fig:G4_simu_data_cluSize,fig:G4_simu_data_Residuals,fig:G4_simu_data_Edep}
comparing data and simulations.

\begin{figure}[htbp] \centering
  \begin{subfigure}[b]{0.23\textwidth}
    \includegraphics[width=\textwidth]{figures/TestBeam/50micron_sizeX.pdf}
    \caption{$50\,\micron$ n-in-p}
  \end{subfigure} \hfill
  \begin{subfigure}[b]{0.23\textwidth}
    \includegraphics[width=\textwidth]{figures/TestBeam/100micron_sizeX.pdf}
    \caption{$100\,\micron$ n-in-p}
  \end{subfigure} \hfill
  \begin{subfigure}[b]{0.23\textwidth}
    \includegraphics[width=\textwidth]{figures/TestBeam/150micron_sizeX.pdf}
    \caption{$150\,\micron$ n-in-p}
  \end{subfigure} \hfill
  \begin{subfigure}[b]{0.23\textwidth}

    \caption{$300\,\micron$ p-in-n}
  \end{subfigure}
  \caption{The cluster size distribution in simulation and data for
    $50\,\micron$, $100\,\micron$, $150\,\micron$ and $300\,\micron$
    thick sensors. The assemblies are operated at the nominal
    conditions.}
  \label{fig:G4_simu_data_cluSize}
\end{figure}

\begin{figure}[htbp] \centering
  \begin{subfigure}[b]{0.23\textwidth}
    \includegraphics[width=\textwidth]{figures/TestBeam/50micron_resX.pdf}
    \caption{$50\,\micron$ n-in-p}
  \end{subfigure} \hfill
  \begin{subfigure}[b]{0.23\textwidth}
    \includegraphics[width=\textwidth]{figures/TestBeam/100micron_resX.pdf}
    \caption{$100\,\micron$ n-in-p}
  \end{subfigure} \hfill
  \begin{subfigure}[b]{0.23\textwidth}
    \includegraphics[width=\textwidth]{figures/TestBeam/150micron_resX.pdf}
    \caption{$150\,\micron$ n-in-p}
  \end{subfigure} \hfill
  \begin{subfigure}[b]{0.23\textwidth}

    \caption{$300\,\micron$ p-in-n}
  \end{subfigure}
  \caption{The residuals distribution in simulation and data for
    $50\,\micron$, $100\,\micron$, $150\,\micron$ and $300\,\micron$
    thick sensors. The assemblies are operated at the nominal
    conditions. The tracking resolution is not unfolded.}
  \label{fig:G4_simu_data_Residuals}
\end{figure}

\begin{figure}[htbp] \centering
  \begin{subfigure}[b]{0.23\textwidth}
    \includegraphics[width=\textwidth]{figures/TestBeam/50micron_Edep.pdf}
    \caption{$50\,\micron$ n-in-p}
  \end{subfigure} \hfill
  \begin{subfigure}[b]{0.23\textwidth}
    \includegraphics[width=\textwidth]{figures/TestBeam/100micron_Edep.pdf}
    \caption{$100\,\micron$ n-in-p}
  \end{subfigure} \hfill
  \begin{subfigure}[b]{0.23\textwidth}
    \includegraphics[width=\textwidth]{figures/TestBeam/150micron_Edep.pdf}
    \caption{$150\,\micron$ n-in-p}
  \end{subfigure} \hfill
  \begin{subfigure}[b]{0.23\textwidth}

    \caption{$300\,\micron$ p-in-n}
  \end{subfigure}
  \caption{The energy deposition in simulation and data for
    $50\,\micron$, $100\,\micron$, $150\,\micron$ and $300\,\micron$
    thick sensors. The assemblies are operated at the nominal
    conditions.}
  \label{fig:G4_simu_data_Edep}
\end{figure}

%% --------------------------------------------- %%
\section{Extrapolation to smaller pixels (CLICpix)}

%% %% --------------------------------------------- %%
%% \begin{figure}[htbp] \centering
%%   \begin{subfigure}[b]{0.45\textwidth}
%%     \includegraphics[width=\textwidth]{./figures/TestBeam/ThresholdScan_W0019_G07.pdf}
%%     \caption{}
%%   \end{subfigure} \hfill
%%   \begin{subfigure}[b]{0.45\textwidth}
%%     \includegraphics[width=\textwidth]{./figures/TestBeam/depletionVoltage_W0019_G07.pdf}
%%     \caption{}
%%   \end{subfigure}
%%   \caption{20-NGR (W19\_G7): bias and voltage scan.}
%%   \label{fig:Timepix3_THLscan_Vdep_G7}
%% \end{figure}

%% \begin{figure}[htbp] \centering
%%   \begin{subfigure}[b]{0.45\textwidth}
%%     \includegraphics[width=\textwidth]{./figures/TestBeam/ThresholdScan_W0019_F07.pdf}
%%     \caption{}
%%   \end{subfigure} \hfill
%%   \begin{subfigure}[b]{0.45\textwidth}
%%     \includegraphics[width=\textwidth]{./figures/TestBeam/depletionVoltage_W0019_F07.pdf}
%%     \caption{}
%%   \end{subfigure}
%%   \caption{23-FGR (W19\_F7): bias and voltage scan.}
%%   \label{fig:Timepix3_THLscan_Vdep_F7}
%% \end{figure}

%% \begin{figure}[htbp] \centering
%%   \begin{subfigure}[b]{0.45\textwidth}
%%     \includegraphics[width=\textwidth]{./figures/TestBeam/ThresholdScan_W0019_L08.pdf}
%%     \caption{}
%%   \end{subfigure} \hfill
%%   \begin{subfigure}[b]{0.45\textwidth}
%%     \includegraphics[width=\textwidth]{./figures/TestBeam/depletionVoltage_W0019_L08.pdf}
%%     \caption{}
%%   \end{subfigure}
%%   \caption{28-GNDGR (W19\_L8): bias and voltage scan.}
%%   \label{fig:Timepix3_THLscan_Vdep_L8}
%% \end{figure}


%% \begin{figure}[htbp] \centering
%%   \begin{subfigure}[b]{0.45\textwidth}
%%     \includegraphics[width=\textwidth]{./figures/TestBeam/ThresholdScan_W0019_C07.pdf}
%%     \caption{}
%%   \end{subfigure} \hfill
%%   \begin{subfigure}[b]{0.45\textwidth}
%%     \includegraphics[width=\textwidth]{./figures/TestBeam/depletionVoltage_W0019_C07.pdf}
%%     \caption{}
%%   \end{subfigure}
%%   \caption{55-GNDGR (W19\_C7): bias and voltage scan.}
%%   \label{fig:Timepix3_THLscan_Vdep_C7}
%% \end{figure}

%% \begin{figure}[htbp] \centering
%%   \begin{subfigure}[b]{0.45\textwidth}
%%     \includegraphics[width=\textwidth]{./figures/TestBeam/ThresholdScan_W0005_E02.pdf}
%%     \caption{}
%%   \end{subfigure} \hfill
%%   \begin{subfigure}[b]{0.45\textwidth}
%%     \includegraphics[width=\textwidth]{./figures/TestBeam/depletionVoltage_W0005_E02.pdf}
%%     \caption{}
%%   \end{subfigure}
%%   \caption{55-GNDGR-100 (W5\_E2): bias and voltage scan.}
%%   \label{fig:Timepix3_THLscan_Vdep_E2}
%% \end{figure}


%% \begin{figure}[htbp] \centering
%%   \begin{subfigure}[b]{0.45\textwidth}
%%     \includegraphics[width=\textwidth]{./figures/TestBeam/ThresholdScan_W0005_F01.pdf}
%%     \caption{}
%%   \end{subfigure} \hfill
%%   \begin{subfigure}[b]{0.45\textwidth}
%%     \includegraphics[width=\textwidth]{./figures/TestBeam/depletionVoltage_W0005_F01.pdf}
%%     \caption{}
%%   \end{subfigure}
%%   \caption{55-GNDGR-150 (W5\_F1): bias and voltage scan.}
%%   \label{fig:Timepix3_THLscan_Vdep_F1}
%% \end{figure}




%% \begin{table}[htbp]
%%   \centering
%%   \caption{Measured depletion voltage for the assemblies described in
%%     \cref{tab:Timepix3Assemblies} and calculated by fitting the
%%     plateau and slope regions of TOT as a function of bias voltage.}
%%   \label{tab:depletionVoltage}
%%   \begin{tabular}{lcccc}
%%     \toprule
%%     Assembly & Thickness [\micron] & Sensor type & Nominal voltage [V] & Depletion voltage [V] \\
%%     \midrule
%%     20-NGR  & 50 & n-in-p & -15 & $<$-9.47 \\
%%     23-FGR & 50 & n-in-p & -15 & $<$-6.32 \\
%%     28-GNDGR & 50 & n-in-p & -15 & $<$-7.19\\
%%     55-GNDGR & 50 & n-in-p & -15 & $<$-5.43\\ \hline
%%     55-GNDGR-100 & 100 & n-in-p & -20 & -10.82 \\ \hline
%%     55-GNDGR-150 & 150 & n-in-p & -30 & -14.86 \\ \hline
%%     W2\_J5       & 300 & p-in-n & 100 & 63.23 \\ 
%%     \bottomrule
%%   \end{tabular}
%% \end{table}


%% \begin{table}[htbp]
%%   \centering
%%   \caption{Measured depletion voltage}
%%   \label{tab:depletionVoltage}
%%   \begin{tabular}{lcccc}
%%     \toprule
%%     Assembly & Thickness [\micron] & Sensor type & Nominal voltage [V] & Depletion voltage [V] \\
%%     \midrule
%%     20-NGR  & 50 & n-in-p & -15 & $<$-9.47 \\
%%     23-FGR & 50 & n-in-p & -15 & $<$-6.32 \\
%%     28-GNDGR & 50 & n-in-p & -15 & $<$-7.19\\
%%     55-GNDGR & 50 & n-in-p & -15 & $<$-5.43\\ \hline
%%     55-GNDGR-100 & 100 & n-in-p & -20 & -10.82 \\ \hline
%%     55-GNDGR-150 & 150 & n-in-p & -30 & -14.86 \\ \hline
%%     W2\_J5       & 300 & p-in-n & 100 & 63.23 \\ 
%%     \bottomrule
%%   \end{tabular}
%% \end{table}


%%%%\subsection{Resolution vs. thickness}
% \begin{figure}[htbp] \centering
%   \begin{subfigure}[b]{0.45\textwidth}
%     \includegraphics[width=\textwidth]{./figures/TestBeam/cluSize_vs_thickness.pdf}
%     \caption{}
%   \end{subfigure} \hfill
%   \begin{subfigure}[b]{0.45\textwidth}
%     \includegraphics[width=\textwidth]{./figures/TestBeam/residuals_vs_thickness.pdf}
%     \caption{}
%   \end{subfigure}
%   \caption{Cluster size and residuals vs. thickness (run 2003 for 300 um thick sensor was taken at 70 V).}
%   \label{fig:clusize_residuals_vs_thickness}
% \end{figure}


%% \begin{figure}[htbp] \centering
%%   \begin{subfigure}[b]{0.3\textwidth}
%%     \includegraphics[width=\textwidth]{figures/TestBeam/50micron_sizeX.pdf}
%%     \caption{}
%%   \end{subfigure} \hfill
%%   \begin{subfigure}[b]{0.3\textwidth}
%%     \includegraphics[width=\textwidth]{figures/TestBeam/50micron_resX.pdf}
%%     \caption{}
%%   \end{subfigure} \hfill
%%   \begin{subfigure}[b]{0.3\textwidth}
%%     \includegraphics[width=\textwidth]{figures/TestBeam/50micron_Edep.pdf}
%%     \caption{}
%%   \end{subfigure}
%%   \caption{For $50\,\micron$ thick sensor.}
%%   \label{fig:G4_simu_data_50micron}
%% \end{figure}

%% \begin{figure}[htbp] \centering
%%   \begin{subfigure}[b]{0.3\textwidth}
%%     \includegraphics[width=\textwidth]{figures/TestBeam/100micron_sizeX.pdf}
%%     \caption{}
%%   \end{subfigure} \hfill
%%   \begin{subfigure}[b]{0.3\textwidth}
%%     \includegraphics[width=\textwidth]{figures/TestBeam/100micron_resX.pdf}
%%     \caption{}
%%   \end{subfigure} \hfill
%%   \begin{subfigure}[b]{0.3\textwidth}
%%     \includegraphics[width=\textwidth]{figures/TestBeam/100micron_Edep.pdf}
%%     \caption{}
%%   \end{subfigure}
%%   \caption{For $100\,\micron$ thick sensor.}
%%   \label{fig:G4_simu_data_100micron}
%% \end{figure}

%% \begin{figure}[htbp] \centering
%%   \begin{subfigure}[b]{0.3\textwidth}
%%     \includegraphics[width=\textwidth]{figures/TestBeam/150micron_sizeX.pdf}
%%     \caption{}
%%   \end{subfigure} \hfill
%%   \begin{subfigure}[b]{0.3\textwidth}
%%     \includegraphics[width=\textwidth]{figures/TestBeam/150micron_resX.pdf}
%%     \caption{}
%%   \end{subfigure} \hfill
%%   \begin{subfigure}[b]{0.3\textwidth}
%%     \includegraphics[width=\textwidth]{figures/TestBeam/150micron_Edep.pdf}
%%     \caption{}
%%   \end{subfigure}
%%   \caption{For $150\,\micron$ thick sensor.}
%%   \label{fig:G4_simu_data_150micron}
%% \end{figure}
