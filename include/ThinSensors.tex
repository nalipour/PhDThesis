% ==============================================================================
\chapter{Thin sensors studies}
\label{ch:ThinSensorsStudies}
% ==============================================================================    

The aim of the CLIC vertex detector is the precise determination of
the displaced vertices and jet flavour-tagging in an environment with
high occupancy from beam-induced backgrounds. Multi-layer barrel and
endcap pixel detectors with geometrical coverage extending down to low
polar angles ($\theta_{min}\sim8^{\circ}$) are foreseen to fulfill
these requirements. The goal is to achieve a single-point resolution
of $\sim 3\,\micron$ with $25\,\micron$ pixel pitch and analog
readout. A material budget of $\sim0.2\%$~X\textsubscript{0} per layer
is required including readout, support and cabling. This can be
achieved with $50\,\micron$ thick sensors on $50\,\micron$ thick
readout ASICs.

As part of the R\&D programme for the CLIC vertex detector, the
performance of thin silicon sensors is quantified using the Timepix3
readout chip with $55\,\micron$ pixel pitch. $50\,\micron$ to
$300\,\micron$ thick planar sensors are bump-bonded to $700\,\micron$
thick Timepix3 ASICs. These assemblies are tested in test beams and
the data are used to compare to \textsc{Geant4}-based simulations. The
simulation is finally extrapolated for smaller pixels of
$25\,\micron$ pixel pitch.

%% --------------------------------------------- %%
\section{Thin-sensor assemblies}
A summary of the Timepix3 assemblies bump-bonded to thin sensors is
shown in \cref{tab:Timepix3Assemblies}. 

The assemblies were tested in beam reference telescope as shown in
\cref{fig:TPX3Telescope}. The assemblies were placed in the middle of
the Timepix3 telescope.

%% --------------------------------------------- %%
\subsection{Operating conditions}\label{sec:operatingConditions}
The Timepix3 readout ASICs were operated with the optimised parameters
as described in \cref{tab:timepix3Operation}. 

The nominal values for the threshold and the bias voltage are shown in
\cref{tab:nominalBiasThreshold}. For the nominal bias voltage, the
sensor is fully (or over) depleted. The nominal threshold set insures
that the readout chip is not operating in noisy conditions. These
nominal values are held throughout the data taking except for the
cases where the threshold or the bias voltages are scanned.

\begin{table}[htbp]
  \centering
  \caption{Nominal operating bias voltage, threshold in DAC and
    calibrated in number of electrons (measured as described in
    \cref{sec:thresholdCalibration}) for the tested assemblies as
    shown in \cref{tab:Timepix3Assemblies}.}
  \label{tab:nominalBiasThreshold}
  \begin{tabular}{lccc}
    \toprule
    Timepix3 ID & Bias voltage [V] & Threshold [DAC] & Threhsold [e\textsuperscript{-}]\\
    \midrule
    W19\_G7 & -15 & 1190 & $526.230\pm0.649$ \\
    W19\_F7 & -15 & 1187 & $600.406\pm0.551$ \\
    W19\_L8 & -15 & 1133 & $568.506\pm0.538$ \\
    W19\_C7 & -15 & 1148 & $608.700\pm0.488$ \\ \hline
    W5\_E2 & -20 & 1160 & $561.019\pm0.712$ \\ \hline
    W5\_F1 & -30 & 1153 & $554.885\pm0.493$ \\ \hline
    W2\_J5 & 100 & 1170 & $565.723\pm1.560$ \\
    \bottomrule
  \end{tabular}
\end{table}

%% --------------------------------------------- %%
\section{Samples and sensors geometries}
%% The DAC settings used for the operation of the Timepix3 assemblies in
%% test beams is summarised in \cref{tab:timepix3Operation}.
%% \begin{itemize}
%% \item I\textsubscript{krum} DAC is set to 10.
%% \item TOT clock frequency: $40\,\megahertz$
%% \item VFBK: 150
%% \end{itemize}


 For most of the measurements, the DUT is perpendicular to the beam. A
 scan was done on the bias voltage and the threshold of the DUT. The
 bias scan allows to obtain the depletion voltage.

%% --------------------------------------------- %%
\section{Experimental results for thin sensors}

\subsection{Measurement of the depletion voltage}

The depletion voltage corresponds to the voltage at which the sensor
is fully depleted and the charge of the full thickness of the sensor
is collected. This can be measured in test beams by scanning the bias
voltage. Lowering the bias voltage causes the under-depletion of the
sensor and therefore the charge collected is also reduced. The
depletion voltage and the depletion width are related by
\cref{eq:depletionVoltage}. 

\cref{fig:W2_J5_DepletionVoltage_chargeVSbiasVoltage} shows the most
probable value (MPV) of the measured TOT as a function of the square
root of the bias voltage. The measured TOT is obtained by summing up
the TOT in the clusters considering all cluster sizes. The
distribution is fitted with a landau function convoluted with a
Gaussian. The fit is done using RooFit~\cite{Cranmer:2012sba} and an
example is given in \cref{fig:W2_J5_DepletionVoltage_TOTdistr}. The
TOT MPV shows two distinct regions: a sloped region and a plateau
region. The intersection of these two regions gives the depletion
voltage of the assembly and it is calculated by the intersection of a
linear fits performed on the both regions. The calculated depletion
voltages for all the assemblies is presented in
\cref{tab:depletionVoltage}.

\begin{figure}[htbp]\centering
  \begin{subfigure}[b]{0.45\textwidth}
  \includegraphics[width=\textwidth]{./figures/TestBeam/W2_J5_totalTOT_Langau_run1998.pdf}
  \caption{}\label{fig:W2_J5_DepletionVoltage_TOTdistr}
  \end{subfigure} \hfill
  \begin{subfigure}[b]{0.45\textwidth}
    \includegraphics[width=\textwidth]{./figures/TestBeam/depletionVoltage_W0002_J05.pdf}
    \caption{}\label{fig:W2_J5_DepletionVoltage_chargeVSbiasVoltage}
  \end{subfigure}
  \caption{(a) The measured TOT distribution in a $300\,\micron$ thick
    silicon sensor (W2\_J5) with a bias voltage of 95~V fitted with a
    Landau function convoluted with a Gaussian function. (b) The most
    probable value of the measured TOT as a function of the bias
    voltage for W2\_J5. Straight lines are used to fit the slope and
    the plateau regions. The depletion voltage corresponds to the
    intersection of these two regions and shown in a red dashed
    line. The continuous red line shows the nominal operating bias
    voltage.}
  \label{fig:W2_J5_DepletionVoltage}
\end{figure}

\begin{table}[htbp]
  \centering
  \caption{Measured depletion voltage for the assemblies described in
    \cref{tab:Timepix3Assemblies} and calculated by fitting the
    plateau and slope regions of TOT as a function of bias voltage.}
  \label{tab:depletionVoltage}
  \begin{tabular}{lc}
    \toprule
    Assembly & Depletion voltage [V] \\
    \midrule
    W19\_G7 & $<$-9.47 \\
    W19\_F7 & $<$-6.32 \\
    W19\_L8 & $<$-7.19\\
    W19\_C7 & $<$-5.43\\ \hline
    W5\_E2 & -10.82 \\ \hline
    W5\_F1 & -14.86 \\ \hline
    W2\_J5 & 63.23 \\ 
    \bottomrule
  \end{tabular}
\end{table}

%% \begin{table}[htbp]
%%   \centering
%%   \caption{Measured depletion voltage for the assemblies described in
%%     \cref{tab:Timepix3Assemblies} and calculated by fitting the
%%     plateau and slope regions of TOT as a function of bias voltage.}
%%   \label{tab:depletionVoltage}
%%   \begin{tabular}{lcccc}
%%     \toprule
%%     Assembly & Thickness [\micron] & Sensor type & Nominal voltage [V] & Depletion voltage [V] \\
%%     \midrule
%%     20-NGR  & 50 & n-in-p & -15 & $<$-9.47 \\
%%     23-FGR & 50 & n-in-p & -15 & $<$-6.32 \\
%%     28-GNDGR & 50 & n-in-p & -15 & $<$-7.19\\
%%     55-GNDGR & 50 & n-in-p & -15 & $<$-5.43\\ \hline
%%     55-GNDGR-100 & 100 & n-in-p & -20 & -10.82 \\ \hline
%%     55-GNDGR-150 & 150 & n-in-p & -30 & -14.86 \\ \hline
%%     W2\_J5       & 300 & p-in-n & 100 & 63.23 \\ 
%%     \bottomrule
%%   \end{tabular}
%% \end{table}


%% \begin{table}[htbp]
%%   \centering
%%   \caption{Measured depletion voltage}
%%   \label{tab:depletionVoltage}
%%   \begin{tabular}{lcccc}
%%     \toprule
%%     Assembly & Thickness [\micron] & Sensor type & Nominal voltage [V] & Depletion voltage [V] \\
%%     \midrule
%%     20-NGR  & 50 & n-in-p & -15 & $<$-9.47 \\
%%     23-FGR & 50 & n-in-p & -15 & $<$-6.32 \\
%%     28-GNDGR & 50 & n-in-p & -15 & $<$-7.19\\
%%     55-GNDGR & 50 & n-in-p & -15 & $<$-5.43\\ \hline
%%     55-GNDGR-100 & 100 & n-in-p & -20 & -10.82 \\ \hline
%%     55-GNDGR-150 & 150 & n-in-p & -30 & -14.86 \\ \hline
%%     W2\_J5       & 300 & p-in-n & 100 & 63.23 \\ 
%%     \bottomrule
%%   \end{tabular}
%% \end{table}


\subsection{Threshold scan}

\subsection{Resolution vs. thickness}
\begin{figure}[htbp] \centering
  \begin{subfigure}[b]{0.45\textwidth}
    \includegraphics[width=\textwidth]{./figures/TestBeam/cluSize_vs_thickness.pdf}
    \caption{}
  \end{subfigure} \hfill
  \begin{subfigure}[b]{0.45\textwidth}
    \includegraphics[width=\textwidth]{./figures/TestBeam/residuals_vs_thickness.pdf}
    \caption{}
  \end{subfigure}
  \caption{Cluster size and residuals vs. thickness (run 2003 for 300 um thick sensor was taken at 70 V).}
  \label{fig:clusize_residuals_vs_thickness}
\end{figure}


\subsection{Efficiency vs. thickness}

The efficiency as a function of the threshold is shown in \cref{fig:efficiency_VS_Threshold}.

\begin{figure}[htbp] 
  \centering
  \includegraphics[width=0.5\textwidth]{./figures/TestBeam/Efficiency_vs_THL.pdf}
  \caption{Efficiency vs. threshold: W2\_J5 is missing.}
  \label{fig:efficiency_VS_Threshold}
\end{figure}


%% --------------------------------------------- %%
\section{Validation of the simulation}

The simulations using AllPix are validated with the data as shown in
\cref{fig:G4_simu_data_50micron,fig:G4_simu_data_100micron,fig:G4_simu_data_150micron}.

\begin{figure}[htbp] \centering
  \begin{subfigure}[b]{0.3\textwidth}
    \includegraphics[width=\textwidth]{figures/TestBeam/50micron_sizeX.pdf}
    \caption{}
  \end{subfigure} \hfill
  \begin{subfigure}[b]{0.3\textwidth}
    \includegraphics[width=\textwidth]{figures/TestBeam/50micron_resX.pdf}
    \caption{}
  \end{subfigure} \hfill
  \begin{subfigure}[b]{0.3\textwidth}
    \includegraphics[width=\textwidth]{figures/TestBeam/50micron_Edep.pdf}
    \caption{}
  \end{subfigure}
  \caption{For $50\,\micron$ thick sensor.}
  \label{fig:G4_simu_data_50micron}
\end{figure}

\begin{figure}[htbp] \centering
  \begin{subfigure}[b]{0.3\textwidth}
    \includegraphics[width=\textwidth]{figures/TestBeam/100micron_sizeX.pdf}
    \caption{}
  \end{subfigure} \hfill
  \begin{subfigure}[b]{0.3\textwidth}
    \includegraphics[width=\textwidth]{figures/TestBeam/100micron_resX.pdf}
    \caption{}
  \end{subfigure} \hfill
  \begin{subfigure}[b]{0.3\textwidth}
    \includegraphics[width=\textwidth]{figures/TestBeam/100micron_Edep.pdf}
    \caption{}
  \end{subfigure}
  \caption{For $100\,\micron$ thick sensor.}
  \label{fig:G4_simu_data_100micron}
\end{figure}

\begin{figure}[htbp] \centering
  \begin{subfigure}[b]{0.3\textwidth}
    \includegraphics[width=\textwidth]{figures/TestBeam/150micron_sizeX.pdf}
    \caption{}
  \end{subfigure} \hfill
  \begin{subfigure}[b]{0.3\textwidth}
    \includegraphics[width=\textwidth]{figures/TestBeam/150micron_resX.pdf}
    \caption{}
  \end{subfigure} \hfill
  \begin{subfigure}[b]{0.3\textwidth}
    \includegraphics[width=\textwidth]{figures/TestBeam/150micron_Edep.pdf}
    \caption{}
  \end{subfigure}
  \caption{For $150\,\micron$ thick sensor.}
  \label{fig:G4_simu_data_150micron}
\end{figure}

%% --------------------------------------------- %%
\section{Extrapolation to smaller pixels (CLICpix)}

%% %% --------------------------------------------- %%
%% \begin{figure}[htbp] \centering
%%   \begin{subfigure}[b]{0.45\textwidth}
%%     \includegraphics[width=\textwidth]{./figures/TestBeam/ThresholdScan_W0019_G07.pdf}
%%     \caption{}
%%   \end{subfigure} \hfill
%%   \begin{subfigure}[b]{0.45\textwidth}
%%     \includegraphics[width=\textwidth]{./figures/TestBeam/depletionVoltage_W0019_G07.pdf}
%%     \caption{}
%%   \end{subfigure}
%%   \caption{20-NGR (W19\_G7): bias and voltage scan.}
%%   \label{fig:Timepix3_THLscan_Vdep_G7}
%% \end{figure}

%% \begin{figure}[htbp] \centering
%%   \begin{subfigure}[b]{0.45\textwidth}
%%     \includegraphics[width=\textwidth]{./figures/TestBeam/ThresholdScan_W0019_F07.pdf}
%%     \caption{}
%%   \end{subfigure} \hfill
%%   \begin{subfigure}[b]{0.45\textwidth}
%%     \includegraphics[width=\textwidth]{./figures/TestBeam/depletionVoltage_W0019_F07.pdf}
%%     \caption{}
%%   \end{subfigure}
%%   \caption{23-FGR (W19\_F7): bias and voltage scan.}
%%   \label{fig:Timepix3_THLscan_Vdep_F7}
%% \end{figure}

%% \begin{figure}[htbp] \centering
%%   \begin{subfigure}[b]{0.45\textwidth}
%%     \includegraphics[width=\textwidth]{./figures/TestBeam/ThresholdScan_W0019_L08.pdf}
%%     \caption{}
%%   \end{subfigure} \hfill
%%   \begin{subfigure}[b]{0.45\textwidth}
%%     \includegraphics[width=\textwidth]{./figures/TestBeam/depletionVoltage_W0019_L08.pdf}
%%     \caption{}
%%   \end{subfigure}
%%   \caption{28-GNDGR (W19\_L8): bias and voltage scan.}
%%   \label{fig:Timepix3_THLscan_Vdep_L8}
%% \end{figure}


%% \begin{figure}[htbp] \centering
%%   \begin{subfigure}[b]{0.45\textwidth}
%%     \includegraphics[width=\textwidth]{./figures/TestBeam/ThresholdScan_W0019_C07.pdf}
%%     \caption{}
%%   \end{subfigure} \hfill
%%   \begin{subfigure}[b]{0.45\textwidth}
%%     \includegraphics[width=\textwidth]{./figures/TestBeam/depletionVoltage_W0019_C07.pdf}
%%     \caption{}
%%   \end{subfigure}
%%   \caption{55-GNDGR (W19\_C7): bias and voltage scan.}
%%   \label{fig:Timepix3_THLscan_Vdep_C7}
%% \end{figure}

%% \begin{figure}[htbp] \centering
%%   \begin{subfigure}[b]{0.45\textwidth}
%%     \includegraphics[width=\textwidth]{./figures/TestBeam/ThresholdScan_W0005_E02.pdf}
%%     \caption{}
%%   \end{subfigure} \hfill
%%   \begin{subfigure}[b]{0.45\textwidth}
%%     \includegraphics[width=\textwidth]{./figures/TestBeam/depletionVoltage_W0005_E02.pdf}
%%     \caption{}
%%   \end{subfigure}
%%   \caption{55-GNDGR-100 (W5\_E2): bias and voltage scan.}
%%   \label{fig:Timepix3_THLscan_Vdep_E2}
%% \end{figure}


%% \begin{figure}[htbp] \centering
%%   \begin{subfigure}[b]{0.45\textwidth}
%%     \includegraphics[width=\textwidth]{./figures/TestBeam/ThresholdScan_W0005_F01.pdf}
%%     \caption{}
%%   \end{subfigure} \hfill
%%   \begin{subfigure}[b]{0.45\textwidth}
%%     \includegraphics[width=\textwidth]{./figures/TestBeam/depletionVoltage_W0005_F01.pdf}
%%     \caption{}
%%   \end{subfigure}
%%   \caption{55-GNDGR-150 (W5\_F1): bias and voltage scan.}
%%   \label{fig:Timepix3_THLscan_Vdep_F1}
%% \end{figure}
