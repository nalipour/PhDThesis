% ==============================================================================
\chapter{The Timepix3 pixel-beam telescope}
\label{ch:Telescope}
%==============================================================================  

%% --------------------------------------------- %% 

Testing in a high energy beam is a crucial step in the R\&D for the
pixel detector sensors and readout chips. Test beam data are used at
various stages of the development for evaluating the performance of a
prototype in addition to simulations tools like TCAD and
\textsc{Geant4} simulations.

Pixel-beam telescopes play a key role in the study of position
sensitive sensors with requirements ranging from high radiation
hardness, high resolution and low material budget but also medical
applications amongst others. A telescope is used to reconstruct the
tracks of the particles going through its planes. The track position
is then extrapolated on the Device Under Test (DUT). This allows to
compare the position of the hit on the DUT with the reconstructed
track and calculate the position and time resolutions and the
efficiency of the device.

For the CLIC vertex detector R\&D, the Timepix3 telescope is used as a
beam reference. Its components, performance and reconstruction
software are described in the following sections. (TO DO: develop more
what is described in the following).

%% --------------------------------------------- %%
\section{Experimental setup at the CERN SPS}
The thin sensors assemblies are tested at the H6 beam of the CERN SPS
using the $120\,\gev$ pion beam. The beam is configured in such a way
to have $\sim2.6 \times 10^6$ particles per spill. The telescope
planes are positioned in a way to give the best tracking resolution
for the given beam energy and the level of multiple scatterings.
%% --------------------------------------------- %%
\section{Components of the Timepix3 telescope}

\subsection{Sensors and mechanics}
The Timepix3 telescope consists of six planes of Timepix3
ASICs~\cite{Timepix3_Poikela} bump bonded to $300\,\micron$ thick
p-in-n planar sensors as shown in \cref{fig:TPX3Telescope}. The planes
are rotated by $9\degrees$ around the x axis (perpendicular to the
beam axis) and the z axis (parallel to the beam axis)
\cite{Akiba:2013yxa}. Given the pixel pitch and the sensor thickness,
this angle mainly leads to clusters of three-hit pixels. Combining the
TOT information, the reconstructed hit position provides sub-pixel
resolution. A tracking resolution of $\sim$$2\,\micron$ on the DUT can
be obtained.


\begin{figure}[htbp]
  \centering
  \begin{tikzpicture}
    \node[anchor=south west,inner sep=0] (image) at
    (0,0){\includegraphics[width=0.6\textwidth]{ActiveEdge/Timepix3Telescope.jpeg}};
    \begin{scope}[x={(image.south east)},y={(image.north west)}]
      \node[above, color=white] at (0.5, 0.85) {Device Under Test};
      \node[above, color=white] at (0.5, 0.78) {(\textbf{DUT})};

      \draw[->, very thick, color=black](0.8, 0.25) -- (0.25, 0.25);
      \node[above, color=black] at (0.5, 0.18) {\textbf{Beam}};
      
      %% \draw[help lines,xstep=.1,ystep=.1] (0, 0) grid (1,1);
      %% \foreach \x in {0,1,...,9} { \node [anchor=north] at (\x/10,0) {0.\x}; }
      %% \foreach \y in {0,1,...,9} { \node [anchor=east] at (0,\y/10) {0.\y}; }
      
    \end{scope}
  \end{tikzpicture} 
  \caption{The Timepix3 beam reference telescope with six planes for
    the tracking and the DUT in the middle inserted perpendicular to
    the beam direction.}
  \label{fig:TPX3Telescope}
\end{figure}

The mechanical support for each telescope plane is minimised in order
to reduce the multiple scatterings of the particles and therefore
improving the tracking resolution. An aluminum support is used to hold
each telescope plane which does not cover the sensor. To protect the
sensors an ABS plastic material is used as cover (in black in
\cref{fig:TPX3Telescope}) with a thickness of 2~mm and placed at a
distance of 10~mm away from the PCB. \cref{tab:TPX3TelescopeMaterial}
describes the material (with their thicknesses and radiation lengths)
seen by the particles for each telescope plane which would contribute
to the multiple scatterings. The PCB stacks up eight layers of copper
interlacing with Isola IS410 type material. An average is done to
estimate the total thickness of Copper in the PCB. Behind the PCB a
layer of Copper is used for the cooling of the chip.

\begin{table}[htbp]
  \centering
  \caption{The material in each telescope plane contributing to the
    multiple scatterings of the particles detected by the sensor where
    X refers to the thickness and X\textsubscript{0} to the radiation
    length (both in millimeters).}
  \label{tab:TPX3TelescopeMaterial}
  \begin{tabular}{l c c c}
    \toprule
    Material & X [mm] & X\textsubscript{0} [mm] & X/X\textsubscript{0} [$\%$] \\
    \midrule
    Cooling material for the chip (Cu) & 0.1 & 14.4 & 0.69 \\
    Isola IS410 in the PCB & 1.475 & 167.6 & 0.88 \\
    Copper in the PCB & 0.125 & 14.4 & 0.87 \\
    ASIC (Si) & 0.7 & 93.7 & 0.75\\
    Sensor (Si) & 0.3 & 93.7 & 0.32\\ 
    Sensor cover (ABS plastic) & 2 & 406.4 & 0.49 \\ \hline
    Total & & & 4 \\
    \bottomrule
  \end{tabular}
\end{table}

\subsection{Coordinates system}
A Cartesian right-handed coordinate system is chosen to describe the
geometry of the telescope. The z-direction is along the beam as shown
in \cref{fig:TPX3Telescope} and the y-direction points vertically in
the up direction.

\subsection{Data acquisition system}
The data-driven zero-suppressed mode (c.f. \cref{sec:TimepixReadout})
is used for the data acquisition of the Timepix3 readout ASICs. 

The SPIDR (Speedy PIxel Detector Readout) read-out system has been
developed by NIKHEF and CERN is used~\cite{Visser:2015bsa}. This
system is able to handle high rates of data sent by the Timepix3 ASIC
at full speed. This is achieved with Xilinx Virtex-7
FPGA~\cite{XilinxVirtex7} and a 10~Gb Ethernet link. This system
allows to record the data from all the particles from the SPS spill
and generates a huge amount of data for processing offline as
described in \cref{sec:recoSoft}.

%% --------------------------------------------- %%
\section{Reconstruction and analysis software frameworks}
\label{sec:recoSoft}

The offline reconstruction of the test beam data is done using two
software frameworks. The EUTelescope software
package~\cite{Rubinskiy,EutelescopeWebsite} is used to reconstruct the
tracks from the telescope and extrapolate their positions on the
DUT. For the analysis of the DUT data, the python-based software
pyEudetanalysis is used (it can be found as a GitHub
directory~\cite{pyeudet}).

\subsection{EUTelescope}
The EUTelescope software is based on the ILCSoft
framework~\cite{Aplin:2009zz}. This latter provides the basic building
blocks such as the LCIO (Linear Collider Input Output) data model, the
geometry description toolkit GEAR and a modular application framework
for event analysis, called Marlin~\cite{Gaede:2006pj}.

A modular analysis and reconstruction chain can be defined using
Marlin processors. Each processor implements algorithms for specific
tasks. The input parameters for the algorithms can be configured and
loaded at runtime using \textit{steering files} in XML format. For
each event, the processors are called centrally by Marlin and this
scheme offers flexibility to the users.

EUTelescope framework, originally developed for the EUDET/AIDA pixel
beam telescope~\cite{Rubinskiy:2014kza}, provides Marlin processors
for the track reconstruction and data analysis of test beam
experiments. \cref{fig:EUTelescope_EUDET_pipeline} schematically shows
the analysis strategy in EUTelescope framework starting from the raw
data recorded during the test beam to finally particle tracks
reconstruction. Since each experiment has its own data format for the
DUT, the format converter is defined by the user. This makes the
framework more flexible for testing different chip families.

\begin{figure}[htbp]
  \centering
  \begin{tikzpicture}
    \node[anchor=south west,inner sep=0] (image) at
    (0,0){\includegraphics[width=0.8\textwidth]{figures/Telescope/EUTelescope_pipeline.png}};
    \begin{scope}[x={(image.south east)},y={(image.north west)}]
    \end{scope}
  \end{tikzpicture} 
  \caption{Data reconstruction and analysis strategy using the
    EUTelescope framework. From~\cite{Jansen:2016bkd}.}
  \label{fig:EUTelescope_EUDET_pipeline}
\end{figure}

The EUTelescope framework is adapted for the reconstruction of the
data from the Timepix3 telescope. The framework is originally written
for a frame-based readout mode and had to be adapted to the
data-driven readout mode which was used during the data taking. This
affects mainly the definition of an event. In a frame-based mode, an
event corresponds to a frame where the shutter of the detectors are
opened and all the hits are integrated during an exposure time. When
the shutter is closed, there is a dead-time where no data is acquired
and all the hits are read out from the readout chips and recorded on
tape in a raw data file. Whereas in the data-driven mode, the data are
acquired as soon as a pixel is hit (there is no shutter and no
dead-time). The hits are written in the raw file with their time-stamps
coming from a combination of the TOA and the FPGA timing. They are not
ordered in time in the raw file since the readout chip does not send
them in the order they are produced. Some pre-processing is needed to
order the hits by their time of arrival (TOA) and create events.

The reconstruction chain for the Timepix3 telescope is described in
the following steps:

\begin{enumerate}
\item Converter: converts the raw files written by each telescope
  planes and the DUT in a binary format to an LCIO event. The
  data-driven zero-suppressed mode is used for the data acquisition of
  the Timepix3 readout ASICs. This mode allows for a very low
  dead-time and all the particles at the SPS spill are recorded. Every
  hit is written with a 64-bit time-stamp (\texttt{long long int})
  related to the TOA of the hits and a time-stamp from the FPGA. The
  hits written in the raw file are not necessarily ordered in time. In
  the converter processor, the hits within a timing window of 3~ms are
  read and filled into a vector. The hits in the vector are ordered in
  time according to their time-stamps. An LCIO event is built by
  choosing the hits in the six telescope planes and the DUT with time
  stamps differing by $2.5\,\microsecond$. With this constraint, most
  probably one track per event is obtained and all the hits belong to
  the same track. Hot pixels (with the maximum allowed frequency of
  0.1 i.e. pixels firing at minimum once every 10 events) are as well
  calculated and also the $\eta$-correction~\cite{Belau:1983eh}
  values.
\item Clustering: the clusters in each telescope plane is found.
\item Hit making: reconstruction of the hit position for each cluster
  with the $\eta$-correction method.
\item Alignment: by assuming straight tracks, the alignment processor
  uses Millepede~II algorithm~\cite{Blobel20065} to align the
  telescope planes with respect to each other. It consists of a least
  squares minimisation problem ($\chi^2$minimisation). A proper
  definition of the geometry is important for this step.
\item Track finding: fits the tracks based on the hits on the
  telescope planes by taking into account the multiple scatterings
  (radiation length of the material for the described geometry) and
  the positions of the planes. The \texttt{EUTelTestFitter} algorithm
  is used for the analysis of the test-beam. The tracks are then
  extrapolated on the DUT.
\end{enumerate} 

\subsection{pyEudetAnalysis}
For the analysis of the DUT data, the python-based software
\texttt{pyEudetanalysis} is used. It can be found as a GitHub directory~\cite{pyeudet}.

%% --------------------------------------------- %%
\section{Timepix3 telescope performance}\label{sec:telescopePerformance}

The Timepix3 telescope is simulated using the AllPix simulation
framework (c.f. \cref{sec:AllPix}). This allows for the understanding
of the telescope performance, the tracking resolution and also the
validation of the simulations. Since AllPix returns the Monte Carlo
Truth position (MC position) of the hits, the true tracking resolution
can be obtained by comparing the MC position with the reconstructed
track or hit positions.  

For the simulations, the telescope planes are placed with the same
positions and rotations as done in data. The geometry of the sensors
and the readout chip are defined based on the hardware. The mechanical
support for the telescope planes is also simulated to have multiple
scatterings as realistic as possible based on the material as
described in \cref{tab:TPX3TelescopeMaterial}. \textsc{Geant4}
provides the class \textsc{G4Material} describes the macroscopic
properties of the material used in simulations and contains all the
information relevant to the constituent elements. The density and the
radiation length of the material are the two main properties to be
considered for the simulations. \textsc{Geant4} already provides a
material database for all the standard material such as copper and
silicon. For the Isola IS410, the material G4\_BONE\_COMPACT\_ICRU has
the closest radiation length and density. For the ABS plastic used as
the sensor covers it composition
(C\textsubscript{8}H\textsubscript{8}~C\textsubscript{4}H\textsubscript{6}~C\textsubscript{3}H\textsubscript{3}N)
has been implemented in the simulations to obtain the correct density
and radiation length.


A digitiser for the Timepix3 readout chips bump-bonded to planar
sensors is defined in AllPix. It simulates the silicon physics by
calculating the diffusion at each \textsc{Geant4} step. 

\subsection{Single-hit resolution on the telescope planes}

The reconstructed hit on each telescope plane is compared to the Monte
Carlo Truth position obtained by the \textsc{Geant4}
simulations. \cref{fig:TelPlane0_MC_hit} compares the hit resolution
on the first telescope plane in x and y directions.


\begin{figure}[htbp] \centering
  \begin{subfigure}[b]{0.45\textwidth}
    \includegraphics[width=\textwidth]{figures/Telescope/telescopePlane0_MC_vs_hit_x.pdf}
    \caption{}
  \end{subfigure}\hfill
  \begin{subfigure}[b]{0.45\textwidth}
    \includegraphics[width=\textwidth]{figures/Telescope/telescopePlane0_MC_vs_hit_y.pdf}
    \caption{}
  \end{subfigure}
  \caption{MC hit vs. reconstructed hit position.}
  \label{fig:TelPlane0_MC_hit}
\end{figure}

The residuals as a function of the x\textsubscript{MC} and
y\textsubscript{MC} are shown in \cref{fig:TelPlane0_MC_hit_2D}.

\begin{figure}[htbp] \centering
  \begin{subfigure}[b]{0.45\textwidth}
    \includegraphics[width=\textwidth]{figures/Telescope/telescopePlane0_MC_vs_hit_x_2D.pdf}
    \caption{}
  \end{subfigure}\hfill
  \begin{subfigure}[b]{0.45\textwidth}
    \includegraphics[width=\textwidth]{figures/Telescope/telescopePlane0_MC_vs_hit_y_2D.pdf}
    \caption{}
  \end{subfigure}
  \caption{MC hit vs. reconstructed hit position.}
  \label{fig:TelPlane0_MC_hit_2D}
\end{figure}


The tracking resolution on the DUT ($100\,\micron$ thick sensor) in
the x and y directions is shown in \cref{fig:DUT_MC_track}.
\begin{figure}[htbp] \centering
  \begin{subfigure}[b]{0.45\textwidth}
    \includegraphics[width=\textwidth]{figures/Telescope/Unbiased_trackRes_DUT_x.pdf}
    \caption{}
  \end{subfigure}\hfill
  \begin{subfigure}[b]{0.45\textwidth}
    \includegraphics[width=\textwidth]{figures/Telescope/Unbiased_trackRes_DUT_y.pdf}
    \caption{}
  \end{subfigure}
  \caption{The tracking resolution on the DUT comparing the
    reconstructed track position to the true position of the particles
    coming from \textsc{Geant4}.}
  \label{fig:DUT_MC_track}
\end{figure}


In 2D the tracking resolution on the DUT is shown in \cref{fig:DUT_MC_track_2D}.
\begin{figure}[htbp] \centering
  \begin{subfigure}[b]{0.45\textwidth}
    \includegraphics[width=\textwidth]{figures/Telescope/Unbiased_trackRes_DUT_x_2D.pdf}
    \caption{}
  \end{subfigure}\hfill
  \begin{subfigure}[b]{0.45\textwidth}
    \includegraphics[width=\textwidth]{figures/Telescope/Unbiased_trackRes_DUT_y_2D.pdf}
    \caption{}
  \end{subfigure}
  \caption{Track vs. MC position on the DUT.}
  \label{fig:DUT_MC_track_2D}
\end{figure}

\subsection{Biased residuals on each telescope plane}

In simulation the material for the PCB in simulations is chosen to be
an epoxy (G4\_PLEXIGLASS) with a density of 1.19~g~\inversecmcubic and
a radiation length of X\textsubscript{0}=4.06~g~\inversecmsquared
(3.41~cm).


The cluster size distribution and the collected charge comparing data
and simulations is shown in \cref{fig:TelescopeCluSize_data_simu}.

\begin{figure}[htbp] \centering
  \begin{subfigure}[b]{0.3\textwidth}
    \includegraphics[width=\textwidth]{figures/Telescope/biasedResiduals/clusterSizeX_telescope0_data_simu.pdf}
    \caption{}
  \end{subfigure}\hfill
  \begin{subfigure}[b]{0.3\textwidth}
    \includegraphics[width=\textwidth]{figures/Telescope/biasedResiduals/clusterSizeY_telescope0_data_simu.pdf}
    \caption{}
  \end{subfigure}\hfill
  \begin{subfigure}[b]{0.3\textwidth}
    \includegraphics[width=\textwidth]{figures/Telescope/biasedResiduals/clusterSignal_telescope0_data_simu.pdf}
    \caption{}
  \end{subfigure}
  \caption{data run 661, simulation run 54.}
  \label{fig:TelescopeCluSize_data_simu}
\end{figure}

The biased residuals (RMS) comparing data and simulation in
\cref{fig:telescopeBiasedRMS_data_simu} (the distributions are given
in
\cref{fig:telescope_biasedResiduals_data_X,fig:telescope_biasedResiduals_data_Y}
for data and
\cref{fig:telescope_biasedResiduals_simu_X,fig:telescope_biasedResiduals_simu_Y}
for simulation.
\begin{figure}[htbp] \centering
  \begin{subfigure}[b]{0.45\textwidth}
    \includegraphics[width=\textwidth]{figures/Telescope/biasedResiduals/RMSX_simu_vs_data.pdf}
    \caption{}
  \end{subfigure}\hfill
  \begin{subfigure}[b]{0.45\textwidth}
    \includegraphics[width=\textwidth]{figures/Telescope/biasedResiduals/RMSY_simu_vs_data.pdf}
    \caption{}
  \end{subfigure}
  \caption{data run 661, simulation run 54.}
  \label{fig:telescopeBiasedRMS_data_simu}
\end{figure}

%% --------------------------------------------- %%

\subsection{Beam angle}

For MC beam angle distribution:

\begin{equation}
  \phi=arctan{{x_2-x_{DUT}} \over {z_2-z_{DUT}}} \; ,
  \label{eq:beamAnglePhi}
\end{equation}

\begin{equation}
  \theta=arctan{{y_2-y_{DUT}} \over {z_2-z_{DUT}}} \; ,
  \label{eq:beamAngleTheta}
\end{equation}

\begin{figure}[htbp] \centering
  \begin{subfigure}[b]{0.45\textwidth}
    \includegraphics[width=\textwidth]{./figures/Telescope/MC_trackAnglePhi_planes_302_100.pdf}
    \caption{}
  \end{subfigure}\hfill
  \begin{subfigure}[b]{0.45\textwidth}
    \includegraphics[width=\textwidth]{./figures/Telescope/MC_trackAngleTheta_planes_302_100.pdf}
    \caption{}
  \end{subfigure}
  \caption{Beam angular distribution in \textsc{Geant4} simulations
(comparing the global positions on the second telescope plane and on
the DUT).}
  \label{fig:MCbeamAngleDistr}
\end{figure}
