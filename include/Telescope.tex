% ==============================================================================
\chapter{The Timepix3 pixel-beam telescope}
\label{ch:Telescope}
% ==============================================================================  

%% --------------------------------------------- %% 

Testing in a high energy beam is a crucial step in the R\&D for the
pixel detector sensors and readout chips. Test beam data are used at
various stages of the development for evaluating the performance of a
prototype in addition to simulations tools like TCAD and
\textsc{Geant4} simulations.

Pixel-beam telescopes play a key role in the study of position
sensitive sensors with requirements ranging from high radiation
hardness, high resolution and low material budget but also medical
applications amongst others. A telescope is used to reconstruct the
tracks of the particles going through its planes. The track position
is then extrapolated on the Device Under Test (DUT). This allows to
compare the position of the hit on the DUT with the reconstructed
track and calculate the position and time resolutions and the
efficiency of the device.

For the CLIC vertex detector R\&D, the Timepix3 telescope is used as a
beam reference. Its components, performance and reconstruction
software are described in the following sections. (TO DO: develop more
what is described in the following).

%% --------------------------------------------- %%
\section{Experimental setup at the CERN SPS}
The thin sensors assemblies are tested at the H6 beam of the CERN SPS
using the $120\,\gev$ pion beam. The beam is configured in such a way
to have $\sim2.6 \times 10^6$ particles per spill. The telescope
planes are positioned in a way to give the best tracking resolution
for the given beam energy and the level of multiple scatterings.
%% --------------------------------------------- %%
\section{Components of the Timepix3 telescope}

\subsection{Sensors and mechanics}
The Timepix3 telescope consists of six planes of Timepix3
ASICs~\cite{Timepix3_Poikela} bump bonded to $300\,\micron$ thick
p-in-n planar sensors as shown in \cref{fig:TPX3Telescope}. The planes
are rotated by $9\degrees$ around the x axis (perpendicular to the
beam axis) and the z axis (parallel to the beam axis)
\cite{Akiba:2013yxa}. Given the pixel pitch and the sensor thickness,
this angle mainly leads to clusters of three-hit pixels. Combining the
TOT information, the reconstructed hit position provides sub-pixel
resolution. A tracking resolution of $\sim$$2\,\micron$ on the DUT can
be obtained.


\begin{figure}[htbp]
  \centering
  \begin{tikzpicture}
    \node[anchor=south west,inner sep=0] (image) at
    (0,0){\includegraphics[width=0.6\textwidth]{ActiveEdge/Timepix3Telescope.jpeg}};
    \begin{scope}[x={(image.south east)},y={(image.north west)}]
      \node[above, color=white] at (0.5, 0.85) {Device Under Test};
      \node[above, color=white] at (0.5, 0.78) {(\textbf{DUT})};

      \draw[->, very thick, color=black](0.8, 0.25) -- (0.25, 0.25);
      \node[above, color=black] at (0.5, 0.18) {\textbf{Beam}};
      
      %% \draw[help lines,xstep=.1,ystep=.1] (0, 0) grid (1,1);
      %% \foreach \x in {0,1,...,9} { \node [anchor=north] at (\x/10,0) {0.\x}; }
      %% \foreach \y in {0,1,...,9} { \node [anchor=east] at (0,\y/10) {0.\y}; }
      
    \end{scope}
  \end{tikzpicture} 
  \caption{The Timepix3 beam reference telescope with six planes for
    the tracking and the DUT in the middle inserted perpendicular to
    the beam direction.}
  \label{fig:TPX3Telescope}
\end{figure}

The mechanical support for each telescope plane is minimised in order
to reduce the multiple scatterings of the particles and therefore
improving the tracking resolution. An aluminum support is used to hold
each telescope plane which does not cover the sensor. To protect the
sensors an ABS plastic material is used as cover (in black in
\cref{fig:TPX3Telescope}) with a thickness of 2~mm and placed at a
distance of 10~mm away from the PCB. \cref{tab:TPX3TelescopeMaterial}
describes the material (with their thicknesses and radiation lengths)
seen by the particles for each telescope plane which would contribute
to the multiple scatterings. The PCB stacks up eight layers of copper
interlacing with Isola IS410 type material. An average is done to
estimate the total thickness of Copper in the PCB. Behind the PCB a
layer of Copper is used for the cooling of the chip.

\begin{table}[htbp]
  \centering
  \caption{The material in each telescope plane contributing to the
    multiple scatterings of the particles detected by the sensor where
    X refers to the thickness and X\textsubscript{0} to the radiation
    length (both in millimeters).}
  \label{tab:TPX3TelescopeMaterial}
  \begin{tabular}{l c c c}
    \toprule
    Material & X [mm] & X\textsubscript{0} [mm] & X/X\textsubscript{0} [$\%$] \\
    \midrule
    Cooling material for the chip (Cu) & 0.1 & 14.4 & 0.69 \\
    Isola IS410 in the PCB & 1.475 & 167.6 & 0.88 \\
    Copper in the PCB & 0.125 & 14.4 & 0.87 \\
    ASIC (Si) & 0.7 & 93.7 & 0.75\\
    Sensor (Si) & 0.3 & 93.7 & 0.32\\ 
    Sensor cover (ABS plastic) & 2 & 406.4 & 0.49 \\ \hline
    Total & & & 4 \\
    \bottomrule
  \end{tabular}
\end{table}

\subsection{Coordinates system}
A Cartesian right-handed coordinate system is chosen to describe the
geometry of the telescope. The z-direction is along the beam as shown
in \cref{fig:TPX3Telescope} and the y-direction points vertically in
the up direction.

\subsection{Data acquisition system}
The data-driven zero-suppressed mode (c.f. \cref{sec:TimepixReadout})
is used for the data acquisition of the Timepix3 readout ASICs. 

This system allows to record the data from all the particles from the
SPS spill and generates a huge amount of data for processing offline
as described in \cref{sec:recoSoft}.


%% --------------------------------------------- %%
\section{Timepix3 telescope performance}\label{sec:telescopePerformance}

The Timepix3 telescope is simulated using the AllPix simulation
framework (c.f. \cref{sec:AllPix}). This allows for the understanding
of the telescope performance, the tracking resolution and also the
validation of the simulations. Since AllPix returns the Monte Carlo
Truth position (MC position) of the hits, the true tracking resolution
can be obtained by comparing the MC position with the reconstructed
track or hit positions.  

\subsection{Timepix3 telescope simulation in AllPix}
For the simulations, the telescope planes are placed with the same
positions and rotations as done in data. The geometry of the sensors
and the readout chip are defined based on the hardware. The mechanical
support for the telescope planes is also simulated to have multiple
scatterings as realistic as possible based on the material as
described in \cref{tab:TPX3TelescopeMaterial}. \textsc{Geant4}
provides the class \textsc{G4Material} describes the macroscopic
properties of the material used in simulations and contains all the
information relevant to the constituent elements. The density and the
radiation length of the material are the two main properties to be
considered for the simulations. \textsc{Geant4} already provides a
material database for all the standard material such as copper and
silicon. For the Isola IS410, the material G4\_BONE\_COMPACT\_ICRU has
the closest radiation length and density. For the ABS plastic, used as
the sensor covers, its constituent elements
(C\textsubscript{8}H\textsubscript{8}~C\textsubscript{4}H\textsubscript{6}~C\textsubscript{3}H\textsubscript{3}N)
have been implemented as a \textsc{G4Material} in the simulations to
obtain the correct density and radiation length.


A digitiser for the Timepix3 readout chips bump-bonded to planar
sensors is defined in AllPix. It simulates the silicon physics by
calculating the diffusion at each \textsc{Geant4} step using
\cref{eq:sigmaDiffusion}. The contribution of the generated charge by
diffusion in the hit pixel and all its direct neighbouring pixels is
calculated for each step. All the contributions add-up and finally the
charge in each pixel is calculated. Finally a threshold is applied to
each pixel and the electronic noise of the Timepix3 ASIC also adds up
to the charge in each pixel. And finally, the TOT value is calculated
by applying the calibration to each assembly.

\subsection{Single-hit resolution on the telescope planes}

\cref{tab:AllPixTelescopePlanesParams} shows the parameters used for
the simulation of the telescope planes. The bias and the depletion
voltages, the sensor type and the threshold values are the parameters
defining the charge sharing and the cluster size
distribution. Therefore they define the intrinsic resolution of the
sensors.

\begin{table}[htbp]
  \centering
  \caption{Parameters used for the simulation of the telescope planes in AllPix.}
  \label{tab:AllPixTelescopePlanesParams}
  \begin{tabular}{cccc}
    \toprule
    Bias voltage [V] & Depletion voltage [V] & Sensor type & Threshold [e-] \\
    \midrule
    50 & 30 & p-in-n & 830 \\
    \bottomrule
  \end{tabular}
\end{table}


\cref{fig:TelescopeCluSize_data_simu} compares the cluster size
distribution and the charge of the clusters in data and simulations
for the first telescope plane.

\begin{figure}[htbp] \centering
  \begin{subfigure}[b]{0.3\textwidth}
    \includegraphics[width=\textwidth]{figures/Telescope/biasedResiduals/clusterSizeX_telescope0_data_simu.pdf}
    \caption{}
  \end{subfigure}\hfill
  \begin{subfigure}[b]{0.3\textwidth}
    \includegraphics[width=\textwidth]{figures/Telescope/biasedResiduals/clusterSizeY_telescope0_data_simu.pdf}
    \caption{}
  \end{subfigure}\hfill
  \begin{subfigure}[b]{0.3\textwidth}
    \includegraphics[width=\textwidth]{figures/Telescope/biasedResiduals/clusterSignal_telescope0_data_simu.pdf}
    \caption{}
  \end{subfigure}
  \caption{For the first telescope plane, the cluster-size distribution in the (a) x direction and (b) y direction. (c) shows the sum of the charge in the cluster in units of TOT.} %data run 661, simulation run 54.
  \label{fig:TelescopeCluSize_data_simu}
\end{figure}

The reconstructed hit on each telescope plane is compared to the Monte
Carlo Truth position (x\textsubscript{MC} and y\textsubscript{MC})
obtained by AllPix simulations. \cref{fig:TelPlane0_MC_hit} compares
the hit resolution on the first telescope plane in x and y
directions. These resolutions are in fact the intrinsic resolution of
the telescope planes sensors and depends on their thickness, rotations
and the charge sharing.

\begin{figure}[htbp] \centering
  \begin{subfigure}[b]{0.45\textwidth}
    \includegraphics[width=\textwidth]{figures/Telescope/telescopePlane0_MC_vs_hit_x.pdf}
    \caption{}
  \end{subfigure}\hfill
  \begin{subfigure}[b]{0.45\textwidth}
    \includegraphics[width=\textwidth]{figures/Telescope/telescopePlane0_MC_vs_hit_y.pdf}
    \caption{}
  \end{subfigure}
  \caption{The difference between the reconstructed hit position .}
  \label{fig:TelPlane0_MC_hit}
\end{figure}

The residuals as a function of the x\textsubscript{MC} and
y\textsubscript{MC} are shown in \cref{fig:TelPlane0_MC_hit_2D}. The
intrinsic resolution does not depend on the position of the hit and
stays the same for all positions. This is also used as a check of
coherence in geometry description in the AllPix simulations and the
EUTelescope reconstruction.

\begin{figure}[htbp] \centering
  \begin{subfigure}[b]{0.45\textwidth}
    \includegraphics[width=\textwidth]{figures/Telescope/telescopePlane0_MC_vs_hit_x_2D.pdf}
    \caption{}
  \end{subfigure}\hfill
  \begin{subfigure}[b]{0.45\textwidth}
    \includegraphics[width=\textwidth]{figures/Telescope/telescopePlane0_MC_vs_hit_y_2D.pdf}
    \caption{}
  \end{subfigure}
  \caption{MC hit vs. reconstructed hit position.}
  \label{fig:TelPlane0_MC_hit_2D}
\end{figure}



\subsection{Biased residuals on each telescope plane}

\cref{fig:telescopeBiasedRMS_data_simu} compares the RMS of the biased
residuals obtained by the difference between the measured hit and the
biased fitted track position in simulations and data on the telescope
planes (the original distributions are given in
\cref{fig:telescope_biasedResiduals_data_X,fig:telescope_biasedResiduals_data_Y}
for data and
\cref{fig:telescope_biasedResiduals_simu_X,fig:telescope_biasedResiduals_simu_Y}
for simulation). For this calculation, a cut is applied to remove
$10\%$ of tracks with the highest
$\chi^2$/NDF. \cref{fig:chi2_data_simu} shows the $\chi^2$/NDF in data
(a) and in simulations (b) with a cut applied illustrated in a red
line.
\begin{figure}[htbp] \centering
  \begin{subfigure}[b]{0.45\textwidth}
    \includegraphics[width=\textwidth]{figures/Telescope/biasedResiduals/RMSX_simu_vs_data.pdf}
    \caption{}
  \end{subfigure}\hfill
  \begin{subfigure}[b]{0.45\textwidth}
    \includegraphics[width=\textwidth]{figures/Telescope/biasedResiduals/RMSY_simu_vs_data.pdf}
    \caption{}
  \end{subfigure}
  \caption{The RMS of the biased residuals in the (a) x and (b) y directions
    comparing the data and simulation for the telescope planes. A cut
    is applied to remove the $10\%$ of tracks with highest
    $\chi^2$/NDF.}
  \label{fig:telescopeBiasedRMS_data_simu}
\end{figure}


\begin{figure}[htbp] \centering
  \begin{subfigure}[b]{0.45\textwidth}
    \includegraphics[width=\textwidth]{figures/Telescope/biasedResiduals/chi2_run661.pdf}
    \caption{Data}
  \end{subfigure}\hfill
  \begin{subfigure}[b]{0.45\textwidth}
    \includegraphics[width=\textwidth]{figures/Telescope/biasedResiduals/chi2_run73.pdf}
    \caption{Simulation}
  \end{subfigure}
  \caption{$\chi^2$/NDF distributions for (a) data and (b)
    simulation. A cut is applied to remove the $10\%$ of tracks with
    highest $\chi^2$/NDF (at the position of the red line).}
  \label{fig:chi2_data_simu}
\end{figure}

The width of the biased residuals increases with the distance (in z
direction). In fact, the uncertainty on the deflection angle due to
the multiple scatterings increases with larger distances of the
telescope planes.



\subsection{Tracking resolution on the DUT}
The tracking resolution on the DUT ($100\,\micron$ thick sensor) in
the x and y directions is shown in \cref{fig:DUT_MC_track}. This value
can be only obtained in simulations as the MC truth position is
needed.
\begin{figure}[htbp] \centering
  \begin{subfigure}[b]{0.45\textwidth}
    \includegraphics[width=\textwidth]{figures/Telescope/Unbiased_trackRes_DUT_x.pdf}
    \caption{}
  \end{subfigure}\hfill
  \begin{subfigure}[b]{0.45\textwidth}
    \includegraphics[width=\textwidth]{figures/Telescope/Unbiased_trackRes_DUT_y.pdf}
    \caption{}
  \end{subfigure}
  \caption{The tracking resolution on the DUT comparing the
    reconstructed track position to the true position of the particles
    coming from \textsc{Geant4}.}
  \label{fig:DUT_MC_track}
\end{figure}


In 2D the tracking resolution on the DUT is shown in \cref{fig:DUT_MC_track_2D}.
\begin{figure}[htbp] \centering
  \begin{subfigure}[b]{0.45\textwidth}
    \includegraphics[width=\textwidth]{figures/Telescope/Unbiased_trackRes_DUT_x_2D.pdf}
    \caption{}
  \end{subfigure}\hfill
  \begin{subfigure}[b]{0.45\textwidth}
    \includegraphics[width=\textwidth]{figures/Telescope/Unbiased_trackRes_DUT_y_2D.pdf}
    \caption{}
  \end{subfigure}
  \caption{Track vs. MC position on the DUT.}
  \label{fig:DUT_MC_track_2D}
\end{figure}

\cref{tab:SummaryOfResolutions} summarises the telescope resolutions
obtained in simulations.

\begin{table}[htbp]
  \centering
  \caption{A summary of the calculated telescope performance.}
  \label{tab:SummaryOfResolutions}
  \begin{tabular}{ccc}
    \toprule
    $\sigma$\textsubscript{int} [$\micron$] & $\sigma$\textsubscript{x,DUT} [$\micron$] & $\sigma$\textsubscript{y,DUT} [$\micron$]\\
    \midrule
    2.4 & 1.86 & 1.60 \\
    \bottomrule
  \end{tabular}
\end{table}


%% --------------------------------------------- %%

\subsection{Beam angle}

For MC beam angle distribution:

\begin{equation}
  \phi=arctan{{x_2-x_{DUT}} \over {z_2-z_{DUT}}} \; ,
  \label{eq:beamAnglePhi}
\end{equation}

\begin{equation}
  \theta=arctan{{y_2-y_{DUT}} \over {z_2-z_{DUT}}} \; ,
  \label{eq:beamAngleTheta}
\end{equation}

\begin{figure}[htbp] \centering
  \begin{subfigure}[b]{0.45\textwidth}
    \includegraphics[width=\textwidth]{./figures/Telescope/MC_trackAnglePhi_planes_302_100.pdf}
    \caption{}
  \end{subfigure}\hfill
  \begin{subfigure}[b]{0.45\textwidth}
    \includegraphics[width=\textwidth]{./figures/Telescope/MC_trackAngleTheta_planes_302_100.pdf}
    \caption{}
  \end{subfigure}
  \caption{Beam angular distribution in \textsc{Geant4} simulations
(comparing the global positions on the second telescope plane and on
the DUT).}
  \label{fig:MCbeamAngleDistr}
\end{figure}
