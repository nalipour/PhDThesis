% ==============================================================================
\chapter{Conclusions and outlook}
\label{ch:conclusions}
% ==============================================================================    


The vertex detector plays a key role to fully exploit the physics
potential at CLIC. To meet the challenging demands on the precision
physics, pixel detectors with high spatial resolution and low material
content are required.

The physics goals set challenging requirements on the design of the
CLIC detector. The vertex detector is responsible for the efficient
tagging of heavy quarks through the precise measurement of the
displaced vertices. For achieving high efficiencies, a low material
content of $\sim0.2\%$~X\textsubscript{0} per vertex layer and a high
single-point resolution of $3\,\micron$ for pixels are required. The
flavour-tagging performance of the CLIC vertex detector in full
detector simulations, have shown that the material budget of the
detector must be as low as possible to achieve higher flavour-tagging
efficiencies. To attain both requirements on the material content and
the resolution, hybrid solutions with $50\,\micron$ thick sensors are
foreseen to be coupled to $50\,\micron$ thick readout ASICs with
$25\,\micron$ pixel pitch.

In this thesis, an overview on the R\&D of planar pixel sensors for
CLIC is given. Thin and active-edge sensors with thicknesses varying
between $50\,\micron$ to $150\,\micron$ are studied. The single point
resolution of thin sensors, with low material content, is being
investigated. Another solution to reduce the material content of the
vertex detector is to prevent overlaps between pixel detectors while
tiling them into ladders. The active-edge technology, allows for
growing the depletion volume up to the physical edges of silicon
sensors and mount them side-by-side and minimise inactive regions. We
span a large set of activities for the characterisation of these
assemblies in test-beam and laboratory measurements. All studies are
accompanied by simulations.

The Timepix3 pixel readout ASIC is chosen as a test-vehicle for thin
and active-edge sensors. The devices under test (DUTs) are integrated
within the CLICdp Timepix3 telescope to obtain a high tracking
resolution. The tracking planes of the telescope also employ the
Timepix3 readout ASIC.

The Timepix3 pixel readout ASIC, with $55\,\micron$ pitch, is a
powerful readout for our studies. With its low noise at the front-end
of $\sim80$ electrons, it allows for operating the chip at the low
thresholds of $\sim500$ electrons. It provides an accurate test-pulse
injection functionality allowing for the calibration of the chip. The
threshold DAC value and the TOT can be therefore converted into energy
depositions.

First, we have investigated on the spatial resolutions achieved with
thin sensors. The DUTs were tested with the CLICdp Timepix3 telescope
during test-beam campaigns at the CERN SPS. The track position allows
for in-pixel studies. The results show that the charge sharing and the
energy deposition in such thin sensors are very low. This results-in
mostly single-pixel clusters. Multi-pixel clusters are generated when
the tracks hit the corners and the edges of the pixels. For a better
hit reconstruction, the energy deposited in the pixels of a cluster
are reconstructed with the $\eta$-correction method and the spatial
resolution can be improved.

For further investigation, the test-beam setup with the devices under
test have been implemented in the \textsc{Geant4}-based simulation
framework, AllPix. A digitiser, describing the physics in silicon
sensors and the Timepix3 readout ASIC, has been developed. The full
test-beam setup with the Timepix3 telescope and the DUTs have been
modeled. First, the tracking resolution of the telescope on the DUT of
$\sim2\,\micron$ is extracted from the simulation. This value can be
obtained only in simulation where the Monte Carlo true position is
available and is in agreement with the expectations on the
tracking. Then, the different assemblies with different thicknesses
have been simulated. A good agreement in data and simulation is seen
between the cluster-size distribution, the position resolution and the
energy deposition spectrum. The charge sharing model as described in
the digitiser of the simulations is validated with data. From the
agreement between the energy deposition in data and simulation we can
conclude that the test-pulse calibration is validated and provides
enough precision for our purposes. The \textsc{Geant4} PAI physics
list to simulate the energy deposition in thin sensors, which provides
a similar spectrum as the Bichsel model, is in agreement with data.

For the CLIC vertex detector, a hit resolution of $\sim3\,\micron$ is
wished to be achieved with pixels of $25\,\micron$ pitch and a
thickness of $50\,\micron$. Since data is still not available for such
pixel detectors, the simulations are extrapolated for such small
pixels. The simulations show that achieving this resolution with
$25\,\micron$ pixels is not possible.


Finally, the efficiency of active-edge sensors is studied. Assemblies
with different edge widths and guard ring configurations are
investigated in test-beam. A 2D TCAD simulation of the edge is as well
implemented which allows for a better understanding of the operation
of such devices.

In simulations, as expected, the floating guard ring seems to be the
best solution. It shows a higher breakdown voltage and a small loss of
energy deposition in the guard ring and therefore a high efficiency up
to the trench of the sensor. The grounded guard ring is however not
recommended for thin sensors of $50\,\micron$ since the guard ring
collects most of the charge deposition at the edge and the edge is
inefficient in this case. But for sensors with higher thicknesses of
$100\,\micron$ and $150\,\micron$, the grounded guard ring is a
suitable solution due to the high energy deposition in such sensors.

In data, the assembly without guard ring has the most efficient edge
among all the assemblies but the breakdown happens at low bias
voltages. The floating guard ring shows an early breakdown (almost
like a grounded guard ring). The edge remains efficient up to the
trench but the signal drops more than what was expected by the
simulations. The grounded guard ring is not efficient for
$50\,\micron$ thick sensors but they become fully efficient for
thicker sensors of $100\,\micron$ and $150\,\micron$.

In general a good agreement was seen between data and simulation for
the different guard ring solutions. The implemented 2D simulation is a
powerful tool to understand the behaviour of the edge in such
technology. However few discrepancies in terms of breakdown behaviour
and the charge collected at the edge between simulations and data are
observed. During the test-beam, we only had one assembly per
configuration at our disposal. More assemblies are needed to be
tested. But we still can conclude that the floating guard ring is the
most suitable solution for $50\,\micron$ thick sensors.

\section{Outlook}

Miniaturisation of pixels to $25\,\micron$ pitch is necessary to
achieve a high spatial resolution of $3\,\micron$ as required by the
CLIC vertex detector. But thin planar sensors of $50\,\micron$, do not
provide enough information through charge sharing in order to reach
the required resolution. Also, bump-bonding of sensors to readout
ASICs becomes more complicated for such small pixel sizes. Other
sensors technologies, such as the High-Voltage (HV) CMOS
process~\cite{Tehrani:2016ogb}, are under investigation for CLIC,
which can be glued to the readout ASICs (instead of bump-bonding), and
provide some information through charge sharing.

Up to now, the full detector simulations for CLIC were done with a
$3\,\micron$ hit resolution for the vertex detector. The extrapolation
done in this thesis for $25\,\micron$ pitch and $50\,\micron$ thick
planar sensors can be taken as input for the digitisers in full
simulations. The effect of a worse resolution on flavour-tagging can
be studied and this might change the requirements on the spatial
resolutions of the CLIC vertex detector.

The two-dimensional TCAD simulation of active-edge sensors gives
promising results and fulfills the expectations. The floating guard
ring appears to be the most suitable solution, as it shows a high
detection efficiency at the edge and an acceptable breakdown
behaviour. In the future, a three-dimensional simulation can provide
more precision on the capacitive coupling between the guard ring and
the pixels. In data, new assemblies are needed to be tested in order
to have more statistics on the results.


%%%-------------------------\\
%%%%%%%%%%%%%%%%%%%%%%%%%%%%%%%%%%%%%%%%%

% \cref{sec:SiliconTheory} gives an introduction to semiconductor
% material with an accent on silicon which is predominant in tracking
% detectors in particle physics experiments. Theoretical concepts are
% introduced and later used for the simulation and understanding of the
% test-beam data. Different methods modeling the energy deposition
% spectrum of high-energy particles in silicon are presented. The
% Bichsel model gives the most accurate energy spectrum for thin
% sensors. The charge transport in a pn-junction is studied. Drift and
% diffusion are calculated in a pn-junction and used in the next
% chapters for simulations. The position measurement and ways to improve
% the spatial resolution are investigated and the crucial role of charge
% sharing in improving the pointing resolution of pixel detectors is
% shown.



% The reconstruction software frameworks used for the analysis of data
% and simulations are described in \cref{ch:Software}. The AllPix
% reconstruction framework and TCAD simulation tools are also
% introduced. Both simulation frameworks are powerfull tools for
% understanding the data. AllPix is a \textsc{Geant4}-based simulation
% frameworks which allows for reproducing the energy-deposition spectrum
% and the charge sharing in thin silicon sensors. The test-beam setup is
% also implemented in this framework and the beam telescope is
% simulated. The finite-element TCAD simulations allow for modeling the
% processing of a silicon detector and the device operation. It is
% mainly used to investigate the performance of the active-edge
% technology in thin silicon sensors.





% \begin{itemize}
% \item Thin planar sensors reaching their limits: other solutions such
%   as HV-CMOS can be also attractive.
% \item Design of readout ASIC important in terms of noise. Operating at
%   lower threshold can help.
% \item Miniaturisation and the limits
% \item Input for realistic digitiser for the CLIC full simulation. 
% \item Active-edge assemblies: floating guard ring is the best solution
%   (breakdown and detection efficiency) for thin sensors. We see an
%   early breakdown and a large fraction of the deposited charge in the
%   guard ring. A readout more adapted for this option is needed to
%   fully confirm the expectations. Ladders of detectors should be built
%   to fully confirm the assumptions.
% \end{itemize}


% Laboratory and test-beam measurements on prototypes
% with thin sensors are presented. \textsc{Geant4}-based and TCAD
% simulations are performed and validated with a comparison to the
% collected data. In this thesis, we span a large set of activities
% starting from calibration of assemblies, to resolution studies of the
% telescope used to finally characterise thin sensors and different
% options of active-edge technology and their impact on the particle
% detection. A summary of these activities is given here below.

% This thesis, gives an overview on the CLIC experiment. The accelerator
% is briefly described and the physics potential of the experiment is
% given. The requirements on the CLIC detector are presented with a
% focus of the vertex detector which is the main subject of the work
% presented.