% ==============================================================================
\chapter{Conclusions}
\label{ch:conclusions}
%==============================================================================    

This thesis, gives an overview on the CLIC experiment. The accelerator
is briefly described and the physics potential of the experiment is
given. The requirements on the CLIC detector are presented with a
focus of the vertex detector which is the main subject of the work
presented. An overview of the R\&D on the planar pixelated silicon
detectors is detailed with laboratory and test-beam
measurements. \textsc{Geant4}-based and TCAD simulations are performed
and validated with a comparison to the collected data. In this thesis,
we span a large set of activities starting from calibration of
assemblies, to resolution studies of the telescope used to finally
characterise thin sensors and different options of active-edge
technology and their impact on the particle detection. A summary of
these activities is given here below.

\cref{sec:SiliconTheory} gives an introduction to semiconductor
material with an accent on silicon which is predominant in tracking
detectors in particle physics experiments. Theoretical concepts are
introduced. Concepts used later in the thesis for the simulation and
understanding of the test-beam data are deeply studied. Different
methods modeling the energy deposition spectrum of high-energy
particles

\cref{ch:FE_electronics}

\cref{ch:Software}

\cref{ch:Telescope}

\cref{ch:ThinSensorsStudies}

\cref{ch:ActiveEdgeSensors}

\section{Outlook}
\begin{itemize}
\item Thin planar sensors reaching their limits: other solutions such
  as HV-CMOS can be also attractive.
\item Design of readout ASIC important in terms of noise. Going to
  lower threshold can help.
\item Miniaturisation and the limits
\item Input for realistic digitiser for the CLIC full simulation
\item Active-edge assemblies: floating guard ring is the best solution
  (breakdown and detection efficiency) for thin sensors. We see an
  early breakdown and a large fraction of the deposited charge in the
  guard ring. A readout more adapted for this option is needed to
  fully confirm the expectations. Ladders of detectors should be built
  to fully confirm the assumptions.
\end{itemize}
