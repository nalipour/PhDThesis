% ==============================================================================
\chapter{Conclusions}
\label{ch:conclusions}
% ==============================================================================    


The vertex detector plays a key role to fully exploit the physics
potential at CLIC. To meet the challenging demands on the precision
physics, pixel detectors with high spatial and timing resolutions with
low material are required.

In this thesis, an overview on the R\&D of planar pixel sensors for
CLIC is given. Prototypes are designed to characterise thin and
active-edge sensors. During this thesis, we span a large set of
activities for this characterisation in test-beam, laboratory
measurements and simulations. A summary is given in the following.

\cref{ch:CLIC} gives an overview on the CLIC experiment. The
accelerator is briefly described and the physics potential of the
experiment is given. The requirements on the CLIC detector are
presented with a focus of the vertex detector which is the main
subject of the work presented.

\cref{sec:SiliconTheory} gives an introduction to semiconductor
material with an accent on silicon which is predominant in tracking
detectors in particle physics experiments. Theoretical concepts are
introduced and later used for the simulation and understanding of the
test-beam data. Different methods modeling the energy deposition
spectrum of high-energy particles are given. The Bichsel model gives
the most accurate energy spectrum for thin sensors. The charge
transport in a pn-junction is studied. Drift and diffusion are
calculated in a pn-junction and used in the next chapters for
simulations. The position measurement and ways to improve the spatial
resolution are investigated.

The Timepix3 pixel readout ASIC is used in the CLICdp Timepix3
telescope as well as for the study of thin and active-edge
sensors. \cref{ch:FE_electronics} introduces this pixel readout
ASIC. The calibration methods to convert the threshold DAC and the TOT
into energy is explained. The calibration is applied to the test-beam
data and the energy spectrum in data agrees well with simulations and
the calibration is therefore validated.

The reconstruction software frameworks used for the analysis of data
and simulations are described in \cref{ch:Software}. The AllPix
reconstruction framework and TCAD simulation tools are explored as
well.

The CLICdp Timepix3 telescope is presented in \cref{ch:Telescope}. It
allows for the reconstruction of the tracks with a high
resolution. The telescope is implemented in \textsc{Geant}-4 based
AllPix simulations. The tracking resolution of $\sim2\,\micron$ is
extracted in simulations. This is in agreement with the expectations.

Test-beam measurements and Allpix simulations for $50-150\,\micron$
thin planar sensors are described in \cref{ch:ThinSensorsStudies}. The
studies are done with a pixel pitch of $55\,\micron$. The simulations
are in good agreement with data. For the CLIC vertex detector a hit
resolution of $\sim3\,\micron$ is wished to be achieved with pixel
detectors of $25\,\micron$ pitch and a thickness of
$50\,\micron$. Since data is still not available for such pixel
detectors, the simulations are extrapolated for such small pixels. The
simulations show that achieving this resolution with $25\,\micron$
pixels is not possible.

\cref{ch:ActiveEdgeSensors} investigates the efficiency of active-edge
sensors. Different edge widths and guard-ring configurations are
investigated. A TCAD simulation of the edge is as well implemented
which allows for better understanding of the operation of such
devices. In simulations, as expected, the floating guard-ring seems to
be the best solution as it shows a higher breakdown voltage and a
small loss of energy deposition in the guard ring and therefore a high
efficiency up to the trench of the sensor. Test-beam data are in
general in good agreement with simulations. However few discrepancies
in terms of breakdown behaviour and the charge collected at the edge
between simulations and data are observed. During the test-beam, we
only had one assembly per configuration at our disposal. More
assemblies are needed to be tested in order to draw final conclusions
on their operation.

Miniaturisation of pixels is necessary for achieving a high spatial
resolution of $3\,\micron$ as required by the CLIC vertex
detector. But combined with thin planar sensors of $50\,\micron$, does
not provide enough information through charge sharing in order to
reach the required resolution. Also, downscaling in a hybrid pixel
detectors makes complications for bump-bonding of sensors to the
readout ASICs. Other sensors solutions with High-Voltage (HV) CMOS
process are considered for CLIC which can be glued to the readout
ASICs (instead of bump-bonding) and also provides information through
charge sharing.

Up to now, the full detector simulations for CLIC were done with a
$3\,\micron$ hit resolution for the vertex detector. The extrapolation
done in this thesis for $25\,\micron$ pitch and $50\,\micron$ planar
sensors can be implemented in full simulations. The effect of a worse
resolution on flavour-tagging can be studied and this might change the
requirements on the CLIC vertex detector.

The two-dimensional TCAD simulation of active-edge sensors gives
promising results and fulfills the expectations. The floating guard
ring appears to be the most suitable solution, as it shows a high
detection efficiency at the edge and an acceptable breakdown
behaviour. A three-dimensional simulation can provide more precision
on the capacitive coupling between the guard ring and the pixels. In
data, new assemblies are needed to be tested in order to chose the
best option of guard ring for thin sensors.

% \begin{itemize}
% \item Thin planar sensors reaching their limits: other solutions such
%   as HV-CMOS can be also attractive.
% \item Design of readout ASIC important in terms of noise. Operating at
%   lower threshold can help.
% \item Miniaturisation and the limits
% \item Input for realistic digitiser for the CLIC full simulation. 
% \item Active-edge assemblies: floating guard ring is the best solution
%   (breakdown and detection efficiency) for thin sensors. We see an
%   early breakdown and a large fraction of the deposited charge in the
%   guard ring. A readout more adapted for this option is needed to
%   fully confirm the expectations. Ladders of detectors should be built
%   to fully confirm the assumptions.
% \end{itemize}


% Laboratory and test-beam measurements on prototypes
% with thin sensors are presented. \textsc{Geant4}-based and TCAD
% simulations are performed and validated with a comparison to the
% collected data. In this thesis, we span a large set of activities
% starting from calibration of assemblies, to resolution studies of the
% telescope used to finally characterise thin sensors and different
% options of active-edge technology and their impact on the particle
% detection. A summary of these activities is given here below.

% This thesis, gives an overview on the CLIC experiment. The accelerator
% is briefly described and the physics potential of the experiment is
% given. The requirements on the CLIC detector are presented with a
% focus of the vertex detector which is the main subject of the work
% presented.