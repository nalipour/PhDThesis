% ==============================================================================
\chapter{Conclusions}
\label{ch:conclusions}
% ==============================================================================    


The vertex detector plays a key role to fully exploit the physics
potential at CLIC. To meet the challenging demands on the precision
physics, pixel detectors with high spatial resolution of
$\sim3\,\micron$ and low material content of
$\sim0.2\%$~X\textsubscript{0} per vertex layer are required. Full
detector simulations have shown that the material budget of the
detector must be as low as possible to achieve a high flavour-tagging
performance. To attain both requirements on the resolution and the
material content, hybrid pixel-detector solutions with $50\,\micron$
thick sensors coupled to $50\,\micron$ thick readout ASICs and
$25\,\micron$ pixel pitch are foreseen.

In the scope of this thesis, the prospects of planar silicon pixel
sensors to meet the challenging requirements in the vertex detector in
terms of spatial resolution and material budget have been
assessed. Thin active-edge sensors with thicknesses varying between
$50\,\micron$ to $150\,\micron$ have been studied in test-beam and
laboratory measurements. All studies were accompanied by simulations.

The high-performance Timepix3 pixel readout ASIC, with $55\,\micron$
pitch, was chosen as a test vehicle for the characterisation of the
sensors. With its low noise at the front-end of $\sim80$ electrons, it
allowed for operating the chip at low thresholds of $\sim500$
electrons. It provided an accurate test-pulse injection functionality
that was used for the energy calibration of the chip and thereby for
an absolute measurement of the energy deposited in the thin sensors.

%% The threshold DAC value and the TOT can be therefore converted into
%% energy depositions.

%% The devices under test (DUTs) are integrated within the
%% CLICdp Timepix3 telescope to obtain a high tracking resolution. The
%% tracking planes of the telescope also employ the Timepix3 readout
%% ASIC.


First, the spatial resolution achievable with thin sensors was
investigated. The devices under test (DUTs) were measured in a
Timepix3-based beam telescope during test-beam campaigns at the CERN
SPS. All of the assemblies have shown an excellent detection
efficiency. In general, a more precise reconstruction of the tracks
passing point was possible for multi-pixel clusters by weighting the
energy deposits in the involved pixels. The $\eta$-algorithm, gave
better results by taking into account the non-linearities due to the
charge sharing. The results however have shown that the amount of
charge sharing in such thin sensors is very low, limiting the
achievable resolution.

For a deeper understanding of the test-beam results, the telescope
setup with the DUTs has been implemented in the \textsc{Geant4}-based
simulation framework, AllPix. A digitiser, describing the charge
transport in planar silicon sensors, has been developed. The tracking
resolution of the telescope on the DUT of $\sim2\,\micron$ has been
extracted from the simulation and validated by comparing the observed
residuals on the telescope planes with the corresponding simulation
results. The validated simulation framework has then been used to
simulate the different assemblies with different thicknesses. A good
agreement between data and simulation was seen for cluster-size
distribution, the position resolution and the energy deposition
spectrum. The charge sharing model as described in the digitiser of
the simulations has been validated with data. The \textsc{Geant4} PAI
physics list to simulate the energy deposition in thin sensors, which
provides a similar spectrum as the Bichsel model, was in agreement
with the data. From the agreement between the energy deposition in
data and simulation it was concluded that the test-pulse calibration
provided enough precision for tracking purposes.

For the CLIC vertex detector, a hit resolution of $\sim3\,\micron$ is
wished to be achieved with pixels of $25\,\micron$ pitch and a sensor
thickness of $50\,\micron$. Since data were not available for such a
configuration, the simulations were extrapolated to small pixels. In
the future, these results could be used as input for the digitiser in
full detector simulations. The results show that achieving high
resolution with such thin sensors is very challenging. New detector
concepts with enhanced charge sharing or even smaller pixel pitch will
therefore be required.


Finally, the efficiency of active-edge sensors were studied. Assemblies
with different edge widths and guard ring configurations were
investigated in test-beam. A two-dimensional TCAD simulation of the
edge was implemented which allowed for a better understanding of the
fabrication process and the operation of such devices.

At the nominal operating conditions, none of the assemblies were
operated beyond the breakdown voltage. Most of the assemblies were
found to be efficient up the physical edge of the sensor. For thin
sensors of $50\,\micron$, the configurations without guard ring or
with floating guard ring were the most suitable designs. A grounded
guard ring was disfavoured due to the loss of efficiency close to the
edge for thin sensors. However, for thicker sensors of $100\,\micron$
and $150\,\micron$, the grounded guard ring was showing high
efficiencies up to the sensor trench due to the thicker silicon bulk.

The two-dimensional TCAD simulations have shown a good description of
the trends and were in agreement with data. The remaining
discrepancies in terms of breakdown behaviour and the charge collected
at the edge between simulations and data are attributed to the
simplifications due to the two-dimensional simulations.


% \section{Outlook}

% Miniaturisation of pixels to $25\,\micron$ pitch is necessary to
% achieve a high spatial resolution of $3\,\micron$ as required by the
% CLIC vertex detector. But thin planar sensors of $50\,\micron$, do not
% provide enough information through charge sharing in order to reach
% the required resolution. Also, bump-bonding of sensors to readout
% ASICs becomes more complicated for such small pixel sizes. Other
% sensors technologies, such as the High-Voltage (HV) CMOS
% process~\cite{Tehrani:2016ogb}, are under investigation for CLIC,
% which can be glued to the readout ASICs (instead of bump-bonding), and
% provide some information through charge sharing.

% Up to now, the full detector simulations for CLIC were done with a
% $3\,\micron$ hit resolution for the vertex detector. The extrapolation
% done in this thesis for $25\,\micron$ pitch and $50\,\micron$ thick
% planar sensors can be taken as input for the digitisers in full
% simulations. The effect of a worse resolution on flavour-tagging can
% be studied and this might change the requirements on the spatial
% resolutions of the CLIC vertex detector.

% The two-dimensional TCAD simulation of active-edge sensors gives
% promising results and fulfills the expectations. The floating guard
% ring appears to be the most suitable solution, as it shows a high
% detection efficiency at the edge and an acceptable breakdown
% behaviour. In the future, a three-dimensional simulation can provide
% more precision on the capacitive coupling between the guard ring and
% the pixels. In data, new assemblies are needed to be tested in order
% to have more statistics on the results.


%%%-------------------------\\
%%%%%%%%%%%%%%%%%%%%%%%%%%%%%%%%%%%%%%%%%

% \cref{sec:SiliconTheory} gives an introduction to semiconductor
% material with an accent on silicon which is predominant in tracking
% detectors in particle physics experiments. Theoretical concepts are
% introduced and later used for the simulation and understanding of the
% test-beam data. Different methods modeling the energy deposition
% spectrum of high-energy particles in silicon are presented. The
% Bichsel model gives the most accurate energy spectrum for thin
% sensors. The charge transport in a pn-junction is studied. Drift and
% diffusion are calculated in a pn-junction and used in the next
% chapters for simulations. The position measurement and ways to improve
% the spatial resolution are investigated and the crucial role of charge
% sharing in improving the pointing resolution of pixel detectors is
% shown.



% The reconstruction software frameworks used for the analysis of data
% and simulations are described in \cref{ch:Software}. The AllPix
% reconstruction framework and TCAD simulation tools are also
% introduced. Both simulation frameworks are powerfull tools for
% understanding the data. AllPix is a \textsc{Geant4}-based simulation
% frameworks which allows for reproducing the energy-deposition spectrum
% and the charge sharing in thin silicon sensors. The test-beam setup is
% also implemented in this framework and the beam telescope is
% simulated. The finite-element TCAD simulations allow for modeling the
% processing of a silicon detector and the device operation. It is
% mainly used to investigate the performance of the active-edge
% technology in thin silicon sensors.





% \begin{itemize}
% \item Thin planar sensors reaching their limits: other solutions such
%   as HV-CMOS can be also attractive.
% \item Design of readout ASIC important in terms of noise. Operating at
%   lower threshold can help.
% \item Miniaturisation and the limits
% \item Input for realistic digitiser for the CLIC full simulation. 
% \item Active-edge assemblies: floating guard ring is the best solution
%   (breakdown and detection efficiency) for thin sensors. We see an
%   early breakdown and a large fraction of the deposited charge in the
%   guard ring. A readout more adapted for this option is needed to
%   fully confirm the expectations. Ladders of detectors should be built
%   to fully confirm the assumptions.
% \end{itemize}


% Laboratory and test-beam measurements on prototypes
% with thin sensors are presented. \textsc{Geant4}-based and TCAD
% simulations are performed and validated with a comparison to the
% collected data. In this thesis, we span a large set of activities
% starting from calibration of assemblies, to resolution studies of the
% telescope used to finally characterise thin sensors and different
% options of active-edge technology and their impact on the particle
% detection. A summary of these activities is given here below.

% This thesis, gives an overview on the CLIC experiment. The accelerator
% is briefly described and the physics potential of the experiment is
% given. The requirements on the CLIC detector are presented with a
% focus of the vertex detector which is the main subject of the work
% presented.
