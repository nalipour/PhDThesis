% ==============================================================================
\chapter{Conclusions}
\label{ch:conclusions}
% ==============================================================================    

The vertex detector plays a key role to fully exploit the physics
potential at CLIC. To meet the challenging demands on the precision
physics, pixel detectors with high spatial and timing resolutions with
low material are required.

In this thesis, an overview on the R\&D of planar pixel sensors for
CLIC is given. Prototypes are designed to characterise thin and
active-edge sensors. During this thesis, we span a large set of
activities for this characterisation in test-beam, laboratory
measurements and simulations. A summary is given in the following.

\cref{sec:SiliconTheory} gives an introduction to semiconductor
material with an accent on silicon which is predominant in tracking
detectors in particle physics experiments. Theoretical concepts are
introduced and later used for the simulation and understanding of the
test-beam data. Different methods modeling the energy deposition
spectrum of high-energy particles are given. The Bichsel model gives
the most accurate energy spectrum for thin sensors. The charge
transport in a pn-junction is studied. Drift and diffusion are
calculated in a pn-junction and used in the next chapters for
simulations. The position measurement and ways to improve the spatial
resolution are investigated.

The Timepix3 pixel readout ASIC is used in the CLICdp Timepix3
telescope as well as for the study of thin and active-edge
sensors. \cref{ch:FE_electronics} introduces this hybrid pixel readout
ASICs. The calibration methods to convert the threshold DAC and the
TOT into energy is explained. The calibration is applied to the
test-beam data and the energy spectrum in data agrees well with
simulations and the calibration is therefore validated.

The reconstruction software frameworks used for the analysis of data
and simulations are described in \cref{ch:Software}. The AllPix
reconstruction framework and TCAD simulation tools are explored as
well.

The CLICdp Timepix3 telescope is presented in \cref{ch:Telescope}. It
allows for the reconstruction of the tracks with a high
resolution. The telescope is implemented in \textsc{Geant}-4 based
AllPix simulations. The tracking resolution of $\sim2\,\micron$ is
extracted in simulations. This is in agreement with the expectations.

Test-beam measurements and Allpix simulations for $50-150\,\micron$
thin planar sensors are described in \cref{ch:ThinSensorsStudies}. The
studies are done with a pixel pitch of $55\,\micron$. The simulations
are in good agreement with data therefore they are used to predict the
resolution with smaller pixels of $25\,\micron$ pitch where data is
still not available. This small pitch and $50\,\micron$ thick sensor
is intended to be used for the CLIC vertex detector where a hit
resolution of $\sim3\,\micron$ is wished to be achieved. The
simulations show that achieving this resolution with $25\,\micron$
pixels is not possible.

\cref{ch:ActiveEdgeSensors} investigates the efficiency of active-edge
sensors. Different edge widths and guard-ring configurations are
investigated. A TCAD simulation of the edge is as well implemented. In
simulations, as expected, the floating guard-ring shows a higher
breakdown voltage and a small loss of energy deposition in the guard
ring and therefore a high efficiency up to the trench of the sensor.

\begin{itemize}
\item Thin planar sensors reaching their limits: other solutions such
  as HV-CMOS can be also attractive.
\item Design of readout ASIC important in terms of noise. Operating at
  lower threshold can help.
\item Miniaturisation and the limits
\item Input for realistic digitiser for the CLIC full simulation. 
\item Active-edge assemblies: floating guard ring is the best solution
  (breakdown and detection efficiency) for thin sensors. We see an
  early breakdown and a large fraction of the deposited charge in the
  guard ring. A readout more adapted for this option is needed to
  fully confirm the expectations. Ladders of detectors should be built
  to fully confirm the assumptions.
\end{itemize}


Laboratory and test-beam measurements on prototypes
with thin sensors are presented. \textsc{Geant4}-based and TCAD
simulations are performed and validated with a comparison to the
collected data. In this thesis, we span a large set of activities
starting from calibration of assemblies, to resolution studies of the
telescope used to finally characterise thin sensors and different
options of active-edge technology and their impact on the particle
detection. A summary of these activities is given here below.

This thesis, gives an overview on the CLIC experiment. The accelerator
is briefly described and the physics potential of the experiment is
given. The requirements on the CLIC detector are presented with a
focus of the vertex detector which is the main subject of the work
presented.