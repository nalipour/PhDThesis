% ==============================================================================
\chapter{Active edge sensors}
\label{ch:ActiveEdgeSensors}
%==============================================================================    


Active-edge sensors allow for seamless tiling of pixel sensors of a
vertex detector by depleting the sensors up to their physical
edges. This allows for high coverage without creating overlaps between
the pixel sensors and therefore reduces the material budget in the
detector. This process consists of extending the backside
implantation to the edge.

In this chapter, the fabrication process for active edge sensors is
studied. Planar sensors produced by Advacam~\cite{AdvacamRef} and bump
bonded to Timepix3 ASICs are tested in test beams. The signal
collection and the efficiency on the edge is presented. The test beam
results are compared to TCAD simulations.
%% --------------------------------------------- %%
\section{Introduction}

Active-edge sensors allow for seamless tiling of pixel sensor matrices
in large areas of a vertex detector by depleting them to their
physical edges. This allows for high coverage without creating
overlaps between the pixel sensors and therefore reduces the
material. Thin n-in-p planar sensors with active edges using a Deep
Reactive Ion Etching (DRIE) process have been produced by
Advacam~\cite{AdvacamRef}. They are bump bonded to the Timepix3
readout chips ($55\,\micron$ pixel pitch) and studied in test beams
and simulations. The DRIE process consists of extending the back-side
implantation, as well as the bias voltage, to the edge of the
sensor. The gradient of potential between the edge and the last pixel
can be very high and could lead to a breakdown of the sensor. A guard
ring consists of an n-implant with a metallic contact on top of it
surrounding the pixel matrix close to the edge and thereby smoothening
the potential transition between the edge and the neighbouring
pixels. The guard ring can be kept floating or grounded by connecting
it to the ground of the readout ASIC. Timepix3 ASICs provide an extra
row of bumped pixels allowing to connect the guard ring to ground
potential. Figure~\ref{fig:activeedge} illustrates a cross section of
an active-edge sensor with and without guard ring.


\begin{figure}[htbp]
  \centering
  \begin{subfigure}[b]{0.45\textwidth}
    \begin{tikzpicture}
      % \node[anchor=south west,inner sep=0] (image) at (0,0){\includegraphics[width=\textwidth]{figures/ActiveEdge/schematic.png}};
      \node[anchor=south west,inner sep=0] (image) at
      (0,0){\includegraphics[width=\textwidth]{figures/ActiveEdge/Efield_20_NGR.png}};
      \begin{scope}[x={(image.south east)},y={(image.north west)}]
        \draw[-, dashed, line width=.7pt, color=white](0.1, 0.05) -- (0.1, 0.92);
        \draw[-, dashed, line width=.7pt, color=white](0.54, 0.05) -- (0.54, 0.92);
        
        \draw[<->, line width=.7pt, color=black](0.01, 1) -- (0.16, 1); % edge width
        \node[above, color=black] at (0.05, 1) {edge};
        
        \draw[<->, thick, color=black](0.17, 1) -- (0.43, 1); % n-implant
        \node[above, color=black] at (0.33, 1) {n-implant};

        \node[above, color=white] at (0.3, 0.5) {\textbf{p-substrate}};
        \draw[<->, thick, color=black](0.54, 0.0) -- (0.98, 0.0); % pixel width
        \node[below, color=black] at (0.75, 0.0) {pixel (55 \micron)};
        
        \draw[-, line width=3pt, color=violet](0.0, 0.05) -- (0.98, 0.05); % p+ backside contact
        \node[below, color=violet] at (0.15, 0.0) {p+ backside contact};
        \draw[-, line width=3pt, color=violet](0.0, 0.045) -- (0.0, 0.93); % p+ active-edge contact
        \node[left, color=violet, rotate=90] at (-0.05, 0.7) {p+ active edge};
        \node[left, color=white, rotate=90] at (0.08, 0.9) {\textbf{final pixel edge}};

        % \draw[help lines,xstep=.1,ystep=.1] (0, 0) grid (1,1);
        % \foreach \x in {0,1,...,9} { \node [anchor=north] at (\x/10,0) {0.\x}; }
        % \foreach \y in {0,1,...,9} { \node [anchor=east] at (0,\y/10) {0.\y}; }

      \end{scope}
    \end{tikzpicture}
    \caption{}
  \end{subfigure}\hfill
  \begin{subfigure}[b]{0.45\textwidth}
    \begin{tikzpicture}
      \node[anchor=south west,inner sep=0] (image) at
      (0,0){\includegraphics[width=\textwidth]{figures/ActiveEdge/Efield_23_FGR.png}};
      \begin{scope}[x={(image.south east)},y={(image.north west)}]
        \draw[-, dashed, line width=.7pt, color=white](0.14, 0.05) -- (0.14, 0.92);
        \draw[-, dashed, line width=.7pt, color=white](0.54, 0.05) -- (0.54, 0.92);
        \draw[<->, line width=.7pt, color=black](0.01, 1) -- (0.16, 1); % edge width
        
        \node[above, color=black] at (0.1, 1) {edge};
        \node[above, color=white] at (0.1, 0.75) {\textbf{GR}};

        % \draw[<->, line width=.4pt, color=black](0.17, 0.97) -- (0.47, 0.97); % n-implant
        % \node[above, color=black] at (0.35, 0.97) {\small{n-implant}};
        \node[above, color=white] at (0.3, 0.5) {\small{p-substrate}};
        \draw[<->, line width=.4pt, color=black](0.54, 0.0) -- (0.98, 0.0); % pixel width
        \node[below, color=black] at (0.75, 0.0) {\small{pixel (55 \micron)}};
        
        \draw[-, line width=3pt, color=violet](0.01, 0.05) -- (0.99, 0.05); % p+ backside contact
  %      \node[below, color=violet] at (0.15, 0.0) {\small{p+ backside contact}};
        \draw[-, line width=3pt, color=violet](0.01, 0.045) -- (0.01, 0.95); % p+ active-edge contact
%        \node[left, color=violet, rotate=90] at (-0.05, 0.7) {\small{p+ active edge}};

        % \draw[help lines,xstep=.1,ystep=.1] (0, 0) grid (1,1);
        % \foreach \x in {0,1,...,9} { \node [anchor=north] at (\x/10,0) {0.\x}; }
        % \foreach \y in {0,1,...,9} { \node [anchor=east] at (0,\y/10) {0.\y}; }
      \end{scope}
    \end{tikzpicture}
    \caption{}
  \end{subfigure}
  \caption{Schematic showing the cross section of a sensor with
    active-edge technology. The pixel edges considered in the analysis
    are indicated with dashed lines. The electric field distribution
    is also illustrated. (a) does not contain any guard ring and (b)
    contains a guard ring (GR) in the edge which consists of an
    n-implant with a metallic contact on top of it.}
  \label{fig:activeedge}
\end{figure}

\section{Active-edge assemblies}\label{sec:AEgeometry}

The produced assemblies by Advacam and tested in test beams are listed
in \cref{tab:activeEdgeAssembliesList}. The edge distance is defined
by the distance between the last pixel implant and the physical sensor
edge.

For the $50\,\micron$ thick sensors, 4 edge configurations are
studied: 
\begin{itemize}
\item 20-NGR does not contain any guard-ring in the edge with an edge
  distance of $20\,\micron$.
\item 23-FGR contains a guard ring with a floating potential. The
  edge distance is $23\,\micron$.
\item 28-GNDGR contains a guard ring connected to the ground potential
  with an edge distance of $28\,\micron$.
\item 55-GNDGR contains a guard ring connected to the ground potential
  with an edge distance of $55\,\micron$.
\end{itemize}

For 55-GNDGR-100 ($100\,\micron$ thick sensor) and 55-GNDGR-150
($150\,\micron$ thick sensor), the edge contains a guard ring
connected to the ground potential with an edge distance of
$55\,\micron$.

\begin{table}[htbp]
  \centering
  \caption{Advacam active-edge n-in-p planar pixel sensor assemblies. The edge distance is defined by the distance between the last pixel implant and the physical sensor edge.}
  \label{tab:activeEdgeAssembliesList}
  \begin{tabular}{lccc}
    \toprule
    Assembly & Thickness [\micron] & Edge distance [\micron] & ID \\
    \midrule
    20-NGR  & 50 & 20 & W19\_G7 \\
    23-FGR & 50 & 23 & W19\_F7 \\
    28-GNDGR & 50 & 28 & W19\_L8 \\
    55-GNDGR & 50 & 55 &W19\_C7 \\ \hline
    55-GNDGR-100 & 100 & 55 & W5\_E2  \\ \hline
    55-GNDGR-150 & 150 & 55 & W5\_F1 \\
    \bottomrule
  \end{tabular}
\end{table}

\cref{fig:Layout_guard_ring} shows the layout of the assemblies listed
in \cref{tab:activeEdgeAssembliesList}.

\begin{figure}[htbp]
  \begin{subfigure}[t]{0.5\textwidth}
    \centering
    \begin{tikzpicture}
      \node[anchor=south west,inner sep=0] at
      (0,0){\includegraphics[width=0.8\textwidth]{figures/ActiveEdge/geometry_20NGR.png}};

      \begin{scope}[x={(image.south east)},y={(image.north west)}]
        \draw[blue, thick](0.56, 0.05)--(0.56, 0.65);
        \draw[blue, thick, dashed](0.52, 0.05)--(0.52, 0.65);
        \draw[blue, thick](0.25, 0.05)--(0.25, 0.65);

        \draw[blue, thick](0.25, 0.05)--(0.56, 0.05);
        \draw[blue, thick](0.25, 0.65)--(0.56, 0.65);

        \node[below, color=blue] at (0.25, 0.055) {-0.055 mm};
        \node[below, color=blue] at (0.52, 0.055) {0 mm};

        \draw[<->, blue, thick](0.46, 0.51)--(0.56, 0.51);
        \node[right, color=blue] at (0.55, 0.51) {Edge distance};

      \end{scope}
    \end{tikzpicture}
    \caption{20-NGR}
    \label{fig:Layout20_NGR}
  \end{subfigure}~
  \begin{subfigure}[t]{0.5\textwidth}
    \centering
    \begin{tikzpicture}
      \node[anchor=south west,inner sep=0] at
      (0,0){\includegraphics[width=0.8\textwidth]{figures/ActiveEdge/geometry_23FGR.png}};
    \end{tikzpicture}
    \caption{23-FGR}
    \label{fig:Layout20_FGR}
  \end{subfigure}
  \begin{subfigure}[t]{0.5\textwidth}
    \centering
    \begin{tikzpicture}
      \node[anchor=south west,inner sep=0] at
      (0,0){\includegraphics[width=0.55\textwidth]{figures/ActiveEdge/geometry_28GNDGR.png}};
    \end{tikzpicture}
    \caption{28-GNDGR}
    \label{fig:Layout20_GNDGR}
  \end{subfigure}~
  \begin{subfigure}[t]{0.5\textwidth}
    \centering
    \begin{tikzpicture}
      \node[anchor=south west,inner sep=0] at
      (0,0){\includegraphics[width=0.55\textwidth]{{figures/ActiveEdge/geometry_55GNDGR.png}}};
    \end{tikzpicture}
    \caption{55-GNDGR, 55-GNDGR-100, 55-GNDGR-150}
    \label{fig:Layout50_GNDGR}
  \end{subfigure}~
  \caption{Sensor layouts for different guard-ring solutions for the
    assemblies described in\cref{tab:activeEdgeAssembliesList}. (a)
    shows the convention used in Figures~\ref{fig:20-NGR},
    \ref{fig:20-FGR}, \ref{fig:20-GNDGR} and \ref{fig:50-GNDGR} to
    express the efficiency and the charge distribution at the edge as
    a function of the track position. The border of the last pixel
    before the edge is indicated by a dashed line (at position 0 mm)
    and the physical sensor edge with a continuous line. The edge
    distance is defined by the distance between the last pixel implant
    and the physical sensor edge.}
  \label{fig:Layout_guard_ring}
\end{figure}












Figure~\ref{fig:IVmeasurements} shows the measured leakage current in
the different assemblies as a function of the bias voltage. The
breakdown occurs earlier for the assembly without guard ring (20-NGR)
compared to the other assemblies. For the nominal operation at -15~V,
none of the assemblies are operated beyond the breakdown voltage.


\begin{figure}[htbp]
  \centering
  \includegraphics[width=0.7\textwidth]{figures/ActiveEdge/IVCurve.pdf}
  \caption{IV measurements for the assemblies listed in Table~\ref{tab:activeEdgeAssembliesList}.}
  \label{fig:IVmeasurements}
\end{figure}


The layers in the geometry description are defined in
Figure~\ref{fig:PixelLayout} and explained in
Table~\ref{tab:PixelStackDimensions}.

\begin{figure}[htbp]
  \centering
  \begin{minipage}[t]{.4\textwidth}
    \centering
    \vspace{0pt}
    \includegraphics[width=0.95\textwidth]{figures/ActiveEdge/pixelLayout_withLayers.png}
    \caption{}
    \label{fig:PixelLayout}
  \end{minipage}
  \hfill
  \begin{minipage}[t]{.56\textwidth}
    \centering
    \vspace{0pt}
    \captionof{table}{Layers in the sensor from the gds file
      (Picture from 23-FGR).}
    \label{tab:PixelStackDimensions}
    \begin{tabular}{l c c}
      \toprule
      Layer number & Layer \\
      \midrule
      6 & metal\\
      3 & - \\
      8 & implant \\
      9 & UBM (for thin film lift off metal) (??) \\
      15 & passivation \\
      5 & contact to connect Al to Si \\
      \bottomrule
    \end{tabular}
  \end{minipage}
\end{figure}

Table~\ref{tab:DimensionsForAssemblies} summarises the dimensions of
the implants for the sensors. The edge width is the distance between
the last pixel implant to the physical edge of the sensor. The metal
width is the diameter of the metal for the pixels. The doping width is
the diameter of the pixels implant. The contact width is the diameter
of the contact between silicon and the metal (where the oxide is
etched). The GR offset is the distance between the physical edge of
the sensor and the implant of the GR.

\begin{table}
  \centering
  \captionof{table}{The pixels and guard-ring dimensions for different assemblies}
  \label{tab:DimensionsForAssemblies}
  \begin{tabular}{l c c c c}
    \toprule
    & 20-NGR & 23-FGR & 28-GNDGR & 55-GNDGR \\
    \midrule
    Edge width [\micron] & 20 & 23 & 28 & 55 \\
    Metal width [\micron] & 40 & 36 & 36 & 40 \\
    Doping width [\micron] & 30 & 30 & 30 & 30 \\
    Contact width [\micron] & 15 & 15 & 15 & 15 \\
    GR offset [\micron] & - & 10 & 14.5 & 25 \\
    GR doping width [\micron] & - & 5 & 5 & 5 \\
    GR contact width [\micron] & - & 3 & 3 & 3 \\
    GR metal width [\micron] & - & 7 & 7 & 10 \\
    \bottomrule
  \end{tabular}
\end{table}






For the 50~\micron grounded GR, the dimensions of the pixels are
differente from above.
\captionof{table}{Layers and dimensions from the gds geometry
  (taken from Timepix 20um GR FLOAt and from Timepix 50~\micron grounded GR).}
\label{tab:PixelStackDimensions}
\begin{tabular}{l c c c}
  \toprule
  Layer number & Layer & Diameter (20 float) [\micron] & Diameter (50 GND) [\micron]\\
  \midrule
  6 & metal & 36 & 40 \\
  3 & - & 34.62 & 36 \\
  8 & implant & 30 & 30 \\
  9 & UBM & 25.6 & 25.6 \\
  15 & passivation & 19.5 & 19.5 \\
  5 & contact to connect Al to Si & 15 & 15 \\
  \bottomrule
\end{tabular}








\subsection{Test-beam analysis of the edge}
\begin{figure}[htbp]
  \begin{subfigure}[b]{0.5\linewidth}
    \centering
    \begin{tikzpicture}
      \node[anchor=south west,inner sep=0] (image) at (0,0) {
        \includegraphics[width=\linewidth, page=3]{figures/TestBeam/edge_bcp.pdf}};
    \end{tikzpicture}
    \caption{}
    \label{fig:}
  \end{subfigure}\hfill
  \begin{subfigure}[b]{0.5\linewidth}
    \centering
    \begin{tikzpicture}
      \node[anchor=south west,inner sep=0] (image) at 
      (0,0){\includegraphics[width=\textwidth, page=5]{figures/TestBeam/edge.pdf}};
    \end{tikzpicture}
    \caption{}
    \label{fig:}
  \end{subfigure}
  \caption{(a) Efficiency and (b) charge (TOT) collected as a function of the
    track position for the assembly 20-NGR.}
  \label{fig:20-NGR_eff_TOT}
\end{figure}


\begin{figure}[htbp]
  \begin{subfigure}[b]{0.5\linewidth}
    \centering
    \begin{tikzpicture}
      \node[anchor=south west,inner sep=0] (image) at (0,0)
      {\includegraphics[width=\textwidth, page=6]{figures/TestBeam/edge_bcp.pdf}};
    \end{tikzpicture}
    \caption{}
  \end{subfigure}\hfill
  \begin{subfigure}[b]{0.5\linewidth}
    \centering
    \begin{tikzpicture}
      \node[anchor=south west,inner sep=0] (image) at 
      (0,0){\includegraphics[width=\textwidth, page=8]{figures/TestBeam/edge.pdf}};
    \end{tikzpicture}
    \caption{}
    \label{fig:}
  \end{subfigure}
  \caption{(a) Efficiency and (b) charge (TOT) collected as a function of the
    track position for the assembly 23-FGR.}
  \label{fig:23-FGR_eff_TOT}
\end{figure}


\begin{figure}[htbp]
  \begin{subfigure}[b]{0.5\linewidth}
    \centering
    \begin{tikzpicture}
      \node[anchor=south west,inner sep=0] (image) at (0,0)
      {\includegraphics[width=\textwidth, page=9]{figures/TestBeam/edge_bcp.pdf}};
    \end{tikzpicture}
    \caption{}
  \end{subfigure}~
  \begin{subfigure}[b]{0.5\linewidth}
    \centering
    \begin{tikzpicture}
      \node[anchor=south west,inner sep=0] (image) at 
      (0,0){\includegraphics[width=\textwidth, page=11]{figures/TestBeam/edge.pdf}};
    \end{tikzpicture}
    \caption{}
  \end{subfigure}
  \caption{(a) Efficiency and (b) charge (TOT) collected as a function of the
    track position for the assembly 28-GNDGR.}
  \label{fig:28-GNDGR_eff_TOT}
\end{figure}


\begin{figure}[htbp]
  \begin{subfigure}[b]{0.5\linewidth}
    \centering
    \begin{tikzpicture}
      \node[anchor=south west,inner sep=0] (image) at (0,0)
      {\includegraphics[width=\textwidth, page=12]{figures/TestBeam/edge_bcp.pdf}};
    \end{tikzpicture}
    \caption{}
  \end{subfigure}\hfill
  \begin{subfigure}[b]{0.5\linewidth}
    \centering
    \begin{tikzpicture}
      \node[anchor=south west,inner sep=0] (image) at 
      (0,0){\includegraphics[width=\textwidth,
        page=14]{figures/TestBeam/edge.pdf}};
    \end{tikzpicture}
    \caption{}
  \end{subfigure}
  \caption{(a) Efficiency and (b) charge (TOT) collected as a function of the
    track position for the assembly 55-GNDGR.}
  \label{fig:55-GNDGR_eff_TOT}
\end{figure}

\begin{figure}[htbp]
  \begin{subfigure}[b]{0.5\linewidth}
    \centering
    \begin{tikzpicture}
      \node[anchor=south west,inner sep=0] (image) at (0,0)
      {\includegraphics[width=\textwidth, page=15]{figures/TestBeam/edge_bcp.pdf}};
    \end{tikzpicture}
    \caption{}
  \end{subfigure}\hfill
  \begin{subfigure}[b]{0.5\linewidth}
    \centering
    \begin{tikzpicture}
      \node[anchor=south west,inner sep=0] (image) at 
      (0,0){\includegraphics[width=\textwidth,
        page=17]{figures/TestBeam/edge.pdf}};
    \end{tikzpicture}
    \caption{}
  \end{subfigure}
  \caption{(a) Efficiency and (b) charge (TOT) collected as a function of the
    track position for the assembly 55-GNDGR-100.}
  \label{fig:55-GNDGR-100_eff_TOT}
\end{figure}


\begin{figure}[htbp]
  \begin{subfigure}[b]{0.5\linewidth}
    \centering
    \begin{tikzpicture}
      \node[anchor=south west,inner sep=0] (image) at (0,0)
      {\includegraphics[width=\textwidth, page=18]{figures/TestBeam/edge_bcp.pdf}};
    \end{tikzpicture}
    \caption{}
  \end{subfigure}\hfill
  \begin{subfigure}[b]{0.5\linewidth}
    \centering
    \begin{tikzpicture}
      \node[anchor=south west,inner sep=0] (image) at 
      (0,0){\includegraphics[width=\textwidth,
        page=20]{figures/TestBeam/edge.pdf}};
    \end{tikzpicture}
    \caption{}
  \end{subfigure}
  \caption{(a) Efficiency and (b) charge (TOT) collected as a function of the
    track position for the assembly 55-GNDGR-150.}
  \label{fig:55-GNDGR-150_eff_TOT}
\end{figure}

%% --------------------------------------------- %%
\newpage
\section{TCAD simulations}
\subsection{Process flow for the active-edge designs}

\begin{itemize}
\item Sprocess 1:
  \begin{enumerate}
  \item First the dimensions of the pixels, implants, contacts, metal
    layer are defined.
  \item The meshing is refined at the borders, the implants and also
    based on the concentration using an adaptive meshing using the
    \texttt{refinebox} command.
  \item The silicon region is then defined for the two pixels and the
    edge region and an extra silicon edge which will be etched (to make
    the process more realistic). The silicon is doped with borons (p-type material)
    with the initial resistivity of $\rho=10000 \Omega$cm ($4.41\times
    10^{11}\,\inversecmcubic$). 
  \item A layer of $0.2\,\micron$ thick of Oxide is deposited.
  \item A layer of $0.2\,\micron$ thick of Nitride is deposited.
  \item The silicon is then doped with borons with a concentration of
    $10^{12}\,\inversecmcubic$. From the bias scan of the $150\,\micron$ thick
    silicon, the depletion voltage was obtained at 16~V and confirms the
    doping concentration of the silicon. The implantation done is done
    at the energy of $180\,\kev$.
  \item The nitride is then etched at the positions where the
    implantation is going to be done. First the a mask is put on the
    positions where the Nitride is going to stay. Then the etching is
    done at the implantation positions. Phosphorus (n-type material) is
    implanted with a dose of $10^{15}\,\inversecmcubic$ with an energy
    of $120\,\kev$.
  \item The extra edge is etched to achieve the edge width wanted. First
    the Nitride layer is etched, then the Oxide and finally the silicon
    layer is etched.
  \item The sensor is then flipped and a layer of oxide is deposited on
    the backside with a thickness of $0.04\,\micron$. An implantation is
    done with Boron with a concentration of $10^{15}\,\inversecmcubic$
    with an energy of $60\,\kev$. Then the oxide is etched from the
    backside and the sensor is flipped again to the initial position.
  \end{enumerate}

\item Sprocess 2:
  \begin{enumerate}
  \item The oxide is then etched at the contact positions.
  \item A photoresist is deposited on the top of the sensor with a
thickness of $2\,\micron$.
  \item The meshing of the edge is then refined adaptively depending
on the concentration of the ions and for a thickness of $1\,\micron$.
  \item Borons are implanted to the edge with a concentration of
$10^{15}\,\inversecmcubic$, an energy of $60\,\kev$ and a title of
$15\degrees$C.
  \item The photoresist is then removed (\texttt{strip resist}).
  \item To activate the dopants, the sensor is annealed at a constant
temperature of $940\degrees$C during 240 minutes.
  \item The metal layer is deposited using Aluminium of thickness
$0.8\,\micron$.
  \item The sensor is then flipped and on the back-side, a layer of
aluminium with a thickness of $0.8\,\micron$ for the contact of the
high-voltage is deposited.
  \end{enumerate}
\end{itemize}

Masks limit the etching and the deposition to a certain range of
window and provide the possibility to imitate the lithographic
patterning.

The doping concentration for the different layouts is shown in
Figure~\ref{fig:TCAD_dopingConcentration}.

%% --------------------------------------------- %%
\subsection{Electric field distribution in simulations}
In silicon, the breakdown field occurs for electric field exceeding
$\sim3\cdot10^5$~\voltpercm. In active-edge sensors, since the
back-side implantation as well as the bias voltage are extended to the
edge of the sensor, the gradient of potential between the edge and the
last pixel can be very high. This could lead to a breakdown of the
sensor. In TCAD simulations, the electric field distribution for the
sensors operated at nominal conditions are shown in
\cref{fig:TCAD_Efield2D}. In any case, for the nominal conditions, the
breakdown electric field is never reached and in the laboratory
measurements, this can be seen through the measurement of the leakage
current as shown in \cref{fig:IVmeasurements}.

\begin{figure}[htbp]
  \centering
  \begin{subfigure}[b]{0.5\linewidth}
    \includegraphics[width=\textwidth]{figures/ActiveEdge/Efield_20_NGR.png}
    \caption{20-NGR}
  \end{subfigure}\hfill
  \begin{subfigure}[b]{0.5\linewidth}
    \includegraphics[width=\textwidth]{figures/ActiveEdge/Efield_23_FGR.png}
    \caption{23-FGR}
  \end{subfigure} \\
  \begin{subfigure}[b]{0.5\linewidth}
    \includegraphics[width=\textwidth]{figures/ActiveEdge/Efield_28_GNDGR.png}
    \caption{28-GNDGR}
  \end{subfigure}\hfill
  \begin{subfigure}[b]{0.5\linewidth}
    \includegraphics[width=\textwidth]{figures/ActiveEdge/Efield_55_GNDGR.png}
    \caption{55-GNDGR}
  \end{subfigure} \\
  \begin{subfigure}[b]{0.5\linewidth}
    \includegraphics[width=\textwidth]{figures/ActiveEdge/Efield_55_GNDGR_100.png}
    \caption{55-GNDGR-100}
  \end{subfigure}\hfill
  \begin{subfigure}[b]{0.5\linewidth}
    \includegraphics[width=\textwidth]{figures/ActiveEdge/Efield_55_GNDGR_150.png}
    \caption{55-GNDGR-150}
  \end{subfigure}
  \caption{Electric field distribution in TCAD simulations.}
  \label{fig:TCAD_Efield2D}
\end{figure}

The electric field and the electrostatic potential in TCAD simulations
for a cut close to the n-implants ($0.2\,\micron$ from the sensor
surface) are shown in
\cref{fig:TCAD_Efield_EPotential_sensorSurface}. Position $0\,\micron$
corresponds to the position of the first pixel. At the surface, the
breakdown electric field is never reached for the nominal
conditions. The floating guard-ring (23-FGR) shows a better potential
transition between the edge of the sensor and the first pixel.


\begin{figure}[htbp]
  \centering
  \begin{subfigure}[b]{0.5\linewidth}
    \includegraphics[width=\textwidth]{figures/ActiveEdge/Efiel_cut0_2um.pdf}
    \caption{}
  \end{subfigure}\hfill
  \begin{subfigure}[b]{0.5\linewidth}
    \includegraphics[width=\textwidth]{figures/ActiveEdge/EPotential_cut0_2um.pdf}
    \caption{}
  \end{subfigure}
  \caption{(a) The electric field and (b) the electrostatic potential
    at a distance of $0.2\,\micron$ from the sensor surface. Position
    $0\,\micron$ corresponds to the position of the first pixel.}
  \label{fig:TCAD_Efield_EPotential_sensorSurface}
\end{figure}

%% --------------------------------------------- %%
\subsection{Validation of simulation with data}

The transient simulation of the active edge devices is done by a
charge deposition of $10^{-5}\,\picocoulomb/\micron$ along the
particle track. This corresponds to an energy deposition of
$\sim80\,\text{e-}/\micron$ as expected for the MIP in
silicon. \cref{fig:TCAD_transientSimu} illustrates a MIP traversing
the sensor at a distance of $10\,\micron$ from the left edge. The
electron density 6~ns after the particle hit is illustrated. The
electrodes collect the current generated by the electrons and it is
integrated over 15~ns.

\begin{figure}[htbp]
  \centering
  \begin{tikzpicture}
    \node[anchor=south west,inner sep=0] (image) at
    (0,0){\includegraphics[width=0.7\textwidth]{figures/ActiveEdge/TCAD_transient_23FGR_hitpos_60.png}};
    \begin{scope}[x={(image.south east)},y={(image.north west)}]

      %% \draw[help lines,xstep=.1,ystep=.1] (0, 0) grid (1,1);
      %% \foreach \x in {0,1,...,9} { \node [anchor=north] at (\x/10,0) {0.\x}; }
      %% \foreach \y in {0,1,...,9} { \node [anchor=east] at (0,\y/10) {0.\y}; }

      \draw[-, dashed, very thick] (0.093, 0.05) -- (0.093, 0.96);
    \end{scope}
  \end{tikzpicture}
  \caption{Transient simulation of a particle track traversing the
    sensor at a distance of $10\,\micron$ from the edge (in dashed
    line). The electron density 6~ns after the particle hit is shown.}
  \label{fig:TCAD_transientSimu}
\end{figure}


In simulation, hits are generated at different positions. The charge
collected by each electrode is calculated over the integration
time. \cref{fig:TCAD_vs_data_20_NGR,fig:TCAD_vs_data_23_FGR,fig:TCAD_vs_data_28_GNDGR,fig:TCAD_vs_data_55_GNDGR,fig:TCAD_vs_data_55_GNDGR_100,fig:TCAD_vs_data_55_GNDGR_150}
compare the charge collected as a function of the hit position in data
and TCAD simulations. In data, the charge deposited is plotted versus
the track position given by the Timepix3 telescope (with a resolution
of $\sim2\,\micron$). The charge deposited is obtained by applying the
test-pulse calibrations to the data. For each track position, the
deposited charge is projected as a one-dimensional histogram and
fitted with a Landau function convoluted to a Gaussian. The most
probable value (MPV) of the Landau fit is then compared to the
collected charge obtained in the TCAD simulations. In the TCAD
simulations, a fixed amount of charge is deposited and, unlike the
data, the fluctuations of the deposited charge are not
considered. This can explain the minor discrepancies on the amount of
the collected charge between simulation and data. The border of the
last pixel (at 0~mm) is indicated with a dashed line and the physical
sensor edge is shown in continuous line.


\newpage
\begin{figure}[htbp]
  \centering
  \begin{subfigure}[b]{0.5\linewidth}
    \includegraphics[width=\textwidth]{figures/ActiveEdge/TCAD_data_Edep_20_NGR.pdf}
    \caption{}
  \end{subfigure}\hfill
  \begin{subfigure}[b]{0.5\linewidth}
    \includegraphics[width=\textwidth]{figures/ActiveEdge/TCAD_data_20_NGR.pdf}
    \caption{}
  \end{subfigure}
  \caption{20-NGR}
  \label{fig:TCAD_vs_data_20_NGR}
\end{figure}

\begin{figure}[htbp]
  \centering
  \begin{subfigure}[b]{0.5\linewidth}

    \caption{}
  \end{subfigure}\hfill
  \begin{subfigure}[b]{0.5\linewidth}
    \includegraphics[width=\textwidth]{figures/ActiveEdge/TCAD_data_23_FGR.pdf}
    \caption{}
  \end{subfigure}
  \caption{23-FGR: no calibration is available: TOT is multiplied by 128 to obtain the Edep in electrons.}
  \label{fig:TCAD_vs_data_23_FGR}
\end{figure}


\begin{figure}[htbp]
  \centering
  \begin{subfigure}[b]{0.5\linewidth}
    \includegraphics[width=\textwidth]{figures/ActiveEdge/TCAD_data_Edep_28_GNDGR.pdf}
    \caption{}
  \end{subfigure}\hfill
  \begin{subfigure}[b]{0.5\linewidth}
    \includegraphics[width=\textwidth]{figures/ActiveEdge/TCAD_data_28_GNDGR.pdf}
    \caption{}
  \end{subfigure}
  \caption{28-GNDGR}
  \label{fig:TCAD_vs_data_28_GNDGR}
\end{figure}


\begin{figure}[htbp]
  \centering
  \begin{subfigure}[b]{0.5\linewidth}
    \includegraphics[width=\textwidth]{figures/ActiveEdge/TCAD_data_Edep_55_GNDGR.pdf}
    \caption{}
  \end{subfigure}\hfill
  \begin{subfigure}[b]{0.5\linewidth}
    \includegraphics[width=\textwidth]{figures/ActiveEdge/TCAD_data_55_GNDGR.pdf}
    \caption{}
  \end{subfigure}
  \caption{55-GNDGR}
  \label{fig:TCAD_vs_data_55_GNDGR}
\end{figure}


\begin{figure}[htbp]
  \centering
  \begin{subfigure}[b]{0.5\linewidth}
    \includegraphics[width=\textwidth]{figures/ActiveEdge/TCAD_data_Edep_55_GNDGR_100.pdf}
    \caption{}
  \end{subfigure}\hfill
  \begin{subfigure}[b]{0.5\linewidth}
    \includegraphics[width=\textwidth]{figures/ActiveEdge/TCAD_data_55_GNDGR_100.pdf}
    \caption{}
  \end{subfigure}
  \caption{55-GNDGR-100}
  \label{fig:TCAD_vs_data_55_GNDGR_100}
\end{figure}



\begin{figure}[htbp]
  \centering
  \begin{subfigure}[b]{0.5\linewidth}
    \includegraphics[width=\textwidth]{figures/ActiveEdge/TCAD_data_Edep_55_GNDGR_150.pdf}
    \caption{}
  \end{subfigure}\hfill
  \begin{subfigure}[b]{0.5\linewidth}
    \includegraphics[width=\textwidth]{figures/ActiveEdge/TCAD_data_55_GNDGR_150.pdf}
    \caption{}
  \end{subfigure}
  \caption{55-GNDGR-150}
  \label{fig:TCAD_vs_data_55_GNDGR_150}
\end{figure}





% \begin{figure}[htbp]
%   \centering
%   \begin{minipage}[t]{.4\textwidth}
%     \centering
%     \vspace{0pt}
%     \includegraphics[width=0.95\textwidth]{figures/ActiveEdge/pixelLayout_withLayers.png}
%     \caption{}
%     \label{fig:PixelLayout}
%   \end{minipage}
%   \hfill
%   \begin{minipage}[t]{.56\textwidth}
%     \centering
%     \vspace{0pt}
%     \captionof{table}{Layers and dimensions from the gds geometry
%       (taken from Timepix 20um GR FLOAT).}
%     \label{tab:PixelStackDimensions}
%     \begin{tabular}{l c c}
%       \toprule
%       Layer number & Layer & Diameter [\micron]\\
%       \midrule
%       6 & metal & 36 \\
%       3 & - & 34.62 \\
%       8 & implant & 30 \\
%       9 & UBM (for thin film lift off metal) (??) & 25.6 \\
%       15 & passivation & 19.5 \\
%       5 & contact to connect Al to Si & 15 \\
%       \bottomrule
%     \end{tabular}
%   \end{minipage}
% \end{figure}




% \begin{figure}[htbp]
%   \centering
%   \begin{subfigure}[b]{0.33\textwidth}
%     \centering
%     \fbox{\includegraphics[width=0.95\textwidth]{figures/ActiveEdge/20umEdge_float_GR_withText.png}}
%     \caption{20~\micron edge: Floating guard ring}
%     \label{fig:GuardRingLayout_20_float_GR}
%   \end{subfigure}\hfill
%   \centering
%   \begin{subfigure}[b]{0.33\textwidth}
%     \centering
%     \fbox{\includegraphics[width=0.95\textwidth]{figures/ActiveEdge/20umEdge_GND_GR_withText.png}}
%     \caption{20~\micron edge: GND guard ring}
%     \label{fig:GuardRingLayout_20_GND_GR}
%   \end{subfigure}\hfill
%   \centering
%   \begin{subfigure}[b]{0.33\textwidth}
%     \centering
%     \fbox{\includegraphics[width=0.95\textwidth]{figures/ActiveEdge/50umEdge_GND_GR_withText.png}}
%     \caption{50~\micron edge: GND guard ring}
%     \label{fig:GuardRingLayout_50_GND_GR}
%   \end{subfigure}
%   \label{fig:GuardRingLayout}
% \end{figure}
