% \chapter{Structure of the thesis}


\begin{enumerate}

\item Introduction
\item CLIC: Compact Linear Collider
  \begin{itemize}
  \item Motivation (post-LHC), Accelerator: two-beam acceleration, CLIC detector
  \item Detector concept, Vertex/tracking detectors for CLIC: Goal, requirements,
    design, technical challenges (cooling/mechanics), Beam induced
    backgrounds, Radiation damage in the vertex detector
  \item Include flavour-tagging plots to make a motivation
  \item Detector simulation software (?)
  \end{itemize}
  
\item Semiconductor detectors for radiation detection: theory, TCAD, \textsc{Geant4}
  \begin{itemize}
  \item Silicon structure
  \item Signal formation: doping, pn-junction, the
    reverse-biased diode, pixel (strip) detectors
  \item Charge collection: drift and diffusion, induced charge (Ramo's
    theorem)
  \item Signal acquisition and electronic noise
  \end{itemize}

\item Hybrid pixel detectors and calibration:
  \begin{itemize}
  \item Readout ASICS: Timepix, Timepix3 (CLICpix)
  \item Test-pulse, source calibration 
  \item Energy deposition and comparison to \textsc{Geant4} and Bichsel models.
  \end{itemize}

\item Thin sensors studies
  \begin{itemize}
  \item Samples, sensors geometries
  \item Test-beam setup: EUDET telescope, Timepix3 telescope
  \item Reconstruction software
  \item Simulation setups and validation: \textsc{Geant4} (Allpix), TCAD 
  \item Edge optimisation (comparison to TCAD).
  \end{itemize}


\item Extrapolation of results for smaller pixel sizes: input for the
  new digitisers for physics simulations

\item Flavour-tagging performance (?)
\item RD53 simulation framework (?)
\item Conclusions
  
\end{enumerate}

